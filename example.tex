%% For normal draft builds (figs undisplayed hence fast compile)
%\documentclass[hyperpdf,nobind,draft,oneside]{hepthesis}
%\documentclass[hyperpdf,nobind,draft,twoside]{hepthesis}

%% For short draft builds (breaks citations by necessity)
%\documentclass[hyperpdf,nobind,draft,hidefrontback]{hepthesis}
\documentclass[hyperpdf,oneside,bindnopdf,usenames,dvipsnames,svgnames,table,longbibliography]{hepthesis}

%%For Cambridge soft-bound version \documentclass[hyperpdf,bindnopdf,usenames,dvipsnames,svgnames,table]{hepthesis}
%% For Cambridge hard-bound version (must be one-sided)
%\documentclass[hyperpdf,oneside]{hepthesis}

%% Load special font packages here if you wish
%\usepackage{lmodern}
\usepackage{mathpazo}
\usepackage{multicol}
%\usepackage{euler}
\usepackage{lineno}
\usepackage{relsize}
\usepackage{subcaption}
\usepackage{caption}
\usepackage{verbatim}
\usepackage{atbegshi}
\usepacakge{pdfpages}
%---------------Add line numbers--------------%
%\linenumbers
%---------------Add line numbers--------------%

%% Put package includes etc. into preamble.tex for convenience
\input{preamble}

\usepackage{floatrow}
% Table float box with bottom caption, box width adjusted to content
\newfloatcommand{capbtabbox}{table}[][\FBwidth]
\usepackage{array}

\newenvironment{conditions}
  {\par\vspace{\abovedisplayskip}\noindent\begin{tabular}{>{$}l<{$} @{${}={}$} l}}
  {\end{tabular}\par\vspace{\belowdisplayskip}}

\newcommand{\mc}[2]{\multicolumn{#1}{c}{#2}}
\definecolor{Gray}{gray}{0.85}
\definecolor{LightCyan}{rgb}{0.8,1,1}
\newcolumntype{a}{>{\columncolor{LightCyan}}c}
%% You can set the line spacing this way
%\setallspacing{double}
%% or a section at a time like this
%\setfrontmatterspacing{double}



%% Doc-specific PDF metadata
\makeatletter
\@ifpackageloaded{hyperref}{%
\hypersetup{%
  pdftitle = {$\mu/\pi$ separation using Convolutional Neural Networks for the MicroBooNE Charged Current Inclusive Cross Section Measurement},
  pdfsubject = {Jessica Esquivel's P.hD Thesis},
  pdfkeywords = {MicroBooNE, CC-Inclusive, physics, FNAL, CNN},
  pdfauthor = {\textcopyright\ Jessica Esquivel}
}}{}
\makeatother

%\AtBeginDocument{\AtBeginShipoutNext{\AtBeginShipoutDiscard}}

%% Start the document
\begin{document}

\begin{frontmatter}

  \thispagestyle{empty}
  \let\cleardoublepage\clearpage
\begin{abstract}%[\smaller \thetitle\\ \vspace*{1cm} \smaller {\theauthor}]
  \thispagestyle{empty}
The purpose of this thesis was to use Convolutional Neural Networks (CNN) to separate $\mu'{s}$ and $\pi'{s}$ for use in increasing the acceptance rate of $\mu'{s}$ below the implemented 75cm track length cut in the Charged Current Inclusive (CC-Inclusive) event selection for the CC-Inclusive Cross-Section Measurement. In doing this, we increase acceptance rate for CC-Inclusive events below a specific momentum range.
\end{abstract}
\let\cleardoublepage\clearpage
\newpage

\begin{center}
\vspace*{0.2cm}
\noindent\makebox[\linewidth]{\rule{\textwidth}{0.3pt}}
{\huge %[Update Title]\\
$\mu/\pi$ separation using \\Convolutional Neural Networks for the \\MicroBooNE Charged Current Inclusive \\Cross Section Measurement 
%Muon/Pion separation using Convolutional Neural Networks for the MicroBooNE Charged Current Inclusive Cross Section Measurement 
%Put in your Title\\[-0.3cm]
%in titlepage.tex\\[0.4cm]
}
\noindent\makebox[\linewidth]{\rule{\textwidth}{0.3pt}}\\ % fix spacing here
\vspace{0.5cm}
%by (fix spacing above and at top)\\
\vspace{0.5cm}
\tracked{\large Jessica Nicole Esquivel}\\[6pt]
Bachelor of Science in Electrical Engineering and Applied Physics\\
St. Mary's University\\
San Antonio, TX, USA 2011\\
\vspace{1.3cm}
\tracked{\large DISSERTATION}\\[6pt]
Submitted in partial fulfillment\\
of the requirements for the degree\\
\emph{Doctor of Philosophy in Physics}\\

\ifthenelse{\boolean{isdraft}}{
\vspace{1.5cm}
- * - \tracked{DRAFT \today}~- * -\\[0.5cm]
}
{
\vspace{1.5cm}
}
May, 2018\\
SYRACUSE UNIVERSITY\\ 
Syracuse, New York\\

\end{center}

%\setcounter{page}{2}
\thispagestyle{empty}
%\clearpage
\let\cleardoublepage\clearpage

\vspace*{5cm}
\begin{center}
    Copyright 2018\\
    Jessica Nicole Esquivel\\
    All Rights Reserved\\
    Syracuse University\\
\end{center}
\thispagestyle{empty}
\let\cleardoublepage\clearpage

 
\vspace*{1cm}
\begin{center}
\adjustbox{height=3cm,width=.2\textwidth, trim=0mm 20mm 0mm 0mm,clip}{\includegraphics[scale=1.3]{figs/seal_bw2}}\includegraphics[scale=.3]{figs/uboone.png}\includegraphics[scale=.15]{figs/nsf2.pdf}\\[10pt] % use trim=l b r t,clip=true to get rid of the bottom white space, trim doesn't work in xelatex
%\adjustbox{height=3cm,width=.2\textwidth, trim=0mm 0mm 12mm 80mm,clip}{}\\[10pt] % also trim this one
\end{center}
%\thispagestyle{empty}

\clearpage

%% Define the un-numbered front matter (cover pages, rubrik and table of contents)
  %\end{titlepage% Title
%\titlepage[of Syracuse University]{%
%  A dissertation submitted to  Syracuse University\\ for the degree of Doctor of Philosophy}

%% Abstract
\begin{abstract}%[\smaller \thetitle\\ \vspace*{1cm} \smaller {\theauthor}]
  %\thispagestyle{empty}
The purpose of this thesis was to use Convolutional Neural Networks (CNN) to separate $\mu'{s}$ and $\pi'{s}$ for use in increasing the acceptance rate of $\mu'{s}$ below the implemented 75cm track length cut in the Charged Current Inclusive (CC-Inclusive) event selection for the CC-Inclusive Cross-Section Measurement. In doing this, we increase acceptance rate for CC-Inclusive events below a specific momentum range.
\end{abstract}

%% Declaration
\begin{declaration}
  I dedicate this dissertation to the two important women in my life; My wife and my mom. Both have been there cheering me on giving me strength and love as I worked towards the hardest accomplishment I've ever done.  
  \vspace*{1cm}
  \begin{flushright}
    Jessica Nicole Esquivel 
  \end{flushright}
\end{declaration}


%\vspace*{1cm}
%\begin{center} % use InkScape to edit these, if necessary.
%\setlength\fboxsep{0pt} % use fbox to see the extra space around graphics
%\setlength\fboxrule{0.5pt} % remove when done
%\fbox{\includegraphics{chick}}
%\adjustbox{height=3cm,trim=0mm 20mm 0mm 0mm,clip}{\includegraphics{figs/seal_bw2}}\\[20pt] % use trim=l b r t,clip=true to get rid of the bottom white space, trim doesn't work in xelatex
%\includegraphics[height=3cm]{figs/lhcb-logo2}\\[20pt] % this one is good, no trimming
%\includegraphics[height=3cm]{figs/LHC-logo}\\[10pt]
%\includegraphics[height=3cm]{figs/cern_logo}\\[20pt]  % this one is good, no trimming
%\adjustbox{height=3cm,trim=0mm 0mm 12mm 80mm,clip}{\includegraphics{figs/nsf2.pdf}}\\[10pt] % also trim this one
%\end{center}
%\thispagestyle{empty}

%\clearpage
%% Acknowledgements
\begin{acknowledgements}
  Of the many people who deserve thanks, some are particularly prominent,
  such as my supervisor\dots
\end{acknowledgements}


%% Preface
%\begin{preface}
%  This thesis describes my research on various aspects of the \LHCb
%  particle physics program, centred around the \LHCb detector and \LHC
%  accelerator at \CERN in Geneva.

%  \noindent
%  For this example, I'll just mention \ChapterRef{chap:SomeStuff}
%  and \ChapterRef{chap:MoreStuff}.
%\end{preface}

%% ToC
\addtocontents{toc}{\protect{\pdfbookmark[0]{\contentsname}{toc}}}
\tableofcontents
\listoffigures
\listoftables

%% Strictly optional!
\frontquote{%
  Well-behaved women \\
  seldom make history.}%
  {Laurel Thatcher Ulrich}
%% I don't want a page number on the following blank page either.
\thispagestyle{empty}

\end{frontmatter}
%\setcounter{page}{6}

%% Start the content body of the thesis
\begin{mainmatter}
  %% Actually, more semantic chapter filenames are better, like "chap-bgtheory.tex"
  % perhaps put introduction in the front matter
% and have three chapters of 1) current understanding
%                            2) experimental considerations
%                            3) theoretical motivation
% decide later, do the writing first
\chapter{Introduction}
%\addcontentsline{toc}{chapter}{Introduction}
This thesis will be a description of work done to further increase effiiciency and purity of the charged current inclusive cross section measurement using the MicroBooNE detector. It will also describe the MicroBooNE detector, what neutrinos are, the charged current inclusive cross section measuremnt and it's importance as well as convolutional neural networks and how they can be used in $\mu/\pi$ separation. 
Chapter \ref{ch:neutrinos} will talk about the background of neutrinos and the people and detectors that discovered neutrinos as well as an in depth history of neutrino oscillation and the discovery that neutrinos have mass. 
Chapter \ref{ch:microboone} will discuss the MicroBooNE experiement, specifically,  how Liquid Argon Time Projection Chambers work, the Light Collection System and the Electronic and Readout Trigger systems.
Chapter \ref{ch:beam} will describe the Booster Neutrino Beam sationed at Fermi National Accelerator Lab. 
Chapter \ref{ch:neutrinoID} will discuss the work that was done to detect the first neutrinos seen in the MicroBooNE detector and the software reconstruction efforts required to create an automated neutrino ID filter that was used to find the first neutrinos and then was later expanded on to create the charged current inclusive filter that will be discussed in chapter \ref{ch:meas}
Chapter \ref{ch:cnn} will give a brief description of what Convolutional Neural Networks are and how it will be used for $\mu/\pi$ separation in this selection.
Chapter \ref{ch:cnn_train} will discuss the hardware frameworks and training methods used to train multiple Convolutional Neural Networks for use in the charged current inclusive cross section measurement.
Chapters \ref{ch:cnn_results}, \ref{ch:data} and \ref{ch:selImodcompare} will discuss the results of using Convolutional Neural Networks on monte-carlo and data to sift out charged current inclusive neutrino events. 

  \chapter{Neutrinos}
%\addcontentsline{toc}{chapter}{Neutrinos}

\section{What are Neutrinos}
Neutrinos are one of the fundamental particles which make up the universe. They are also one of the least understood. Neutrinos are not affected by the electromagnetic forces because they do not have electric charge. Neutrinos are affected by a "weak" sub-atomic force of much shorter range than electromagnetism, and are therefore able to pass through great distances in matter without being affected by it. Until the late 90's, neutrinos were thought to have no mass. Due to their mass, neutrinos are also affected by gravity. Neutrinos are created by radioactive decay or nuclear reactions such as the ones that happen in the sun, in nuclear reactors or when cosmic rays hit atoms. There are three types of neutrinos, $\nu_{e}$, $\nu_{\mu}$ and $\nu_{\tau}$ which correspond to their charged lepton pairs.  

As previously stated, neutrinos are very weakly interacting; in fact, neutrinos can pass unscathed through a wall of lead several hundred light-years thick. Because neutrinos interact so rarely, studying neutrinos requires a massive detector and a powerful neutrino source. With that being said, we can only infer their existence when they interact in a detector. In a collision, distinct charged particles are produced with each type of neutrino. An electron neutrino will create an electron, a muon neutrino will create a muon, and a tau neutrino will create a tau. The track the particle leaves in the detector is how one figures out what type of neutrino interaction was “seen.” Liquid Argon Time Projection Chambers are the newest type of detectors being used to study neutrinos due to their excellent imaging and particle identification capabilities. 

\section{History of Neutrinos}
The neutrino was first postulated by Wolfgang Pauli in 1931 to explain how beta decay could resolve the conservation of energy, momentum and angular momentum problem. Pauli suggested that this missing energy might be carried off, unseen, by a neutral particle (he called neutron) which was escaping detection. James Chadwick discovered a much heavier nuclear particle in 1932 that he also named neutron, leaving two particles with the same name. Enrico Fermi was the first person to coin the term neutrino (which means little neutral one in latin) in 1933 to fix this confusion. Fermi's paper, which was published in 1934, unified Pauli's neutrino with Paul Dirac's positron and Werner Heisenberg's neutron-proton model and his theory accurately explained many experimentally observed results. Wang Ganchang first proposed the use of beta capture to experimentally detect neutrinos and in 1959 Clyde Cowan and Frederick Reines published their work stating that they had detected the neutrino. The experiment called for antineutrinos created in a nuclear reactor by beta decay that reacted with protons producing neutrons and positrons: $\nu_{e} +p^{+}\rightarrow n^{0} + e^{+}$. Once this happens, the positron finds an electron and they annihilate each other and the resulting gamma rays are detectable. The neutron is detected by neutron capture and the releasing of another gamma ray. In 1962 Leon M. Lenderman, Melvin Schwartz and Jack Steinberger were the first to detect interactions of the muon neutrino. The first detection of the tau neutrino was announced in the summer of 2000 by the DONUT collaboration at Fermilab. In the late 1960s, many experiments found that the number of electron neutrinos arriving from the sun was around $1/3$ to $1/2$ the number predicted by the Standard Solar Model. This became known as the solar neutrino problem and remained unresolved for around thirty years. This problem was resolved by the discovery of neutrino oscillation and mass.\cite{neutrino}


\section{Neutrino Oscillations}
Neutrino oscillation was first predicted by Bruno Pontecorvo. It describes the phenomenon of a neutrino created with a specific lepton flavour (electron, muon or tau) that is later measured to have a different flavour. Neutrino oscillation is important theoretically and experimentally due to the fact that this observation implies that the neutrino has a non-zero mass, which is not part of the original Standard Model of particle physics. \cite{neutrinooscillation} 

\section{Solar Oscillations and the Solar Neutrino Problem}

The solar neutrino flux derived from Bahcall's Standard Solar Model is shown in figure \ref{fig:solarmodel}. Nuclear fusion and decay processes produce an abundant amount of neutrinos. The standard solar model predicts that these reactions produce several groups of neutrinos, each with differing fluxes and energy spectra. The figure also shows the ranges of detection of existing solar neutrino experiments in different shades of blue to illustrate that they sample different portions of the solar neutrino energy spectrum. Three of these experiments, plus a new one, are discussed below.

\begin{figure}[htp]
\centering
\includegraphics[scale=.55]{figs/solarmodel.jpg}
\captionof{figure}{The Standard Solar Model}
\label{fig:solarmodel}
\end{figure}

Since neutrinos rarely interact with matter, they pass through the sun and the earth undetected. About 65-billion neutrinos from the sun stream through every square centimeter on the Earth every second, yet we are oblivious to their passage in our every-day lives. \cite{bnl}

The first experiment to detect the effects of neutrino oscillation was the Ray Davis's Homestake Experiment. The detector was stationed in the Homestake Gold Mine in Lead, South Dakota. It was 1,478 meters underground and was 380 $m^{3}$. The detector was filled with perchloroethylene. Perchloroethylene was chosen because of it's high concentrations of chlorine. When an $\nu_{e}$ interaced with chlorine-37 atom, the atom would transform to argon-37 which was then extracted and counted. The neutrino capture reaction is shown in equation \ref{eq:capture}. Davis observed a deficit of about $1/3$ the flux of solar neutrinos that was predicted by Bahcall's Standard Solar Model. The unexplained difference between the measured solar neutrino flux and model predictions lead to the Solar Neutrino Problem.\cite{fnal}

\begin{equation}
\label{eq:capture}
\nu_{e} +{}^{37}Cl \rightarrow {}^{37}Ar + e^{-}
\end{equation}

While it is now known that the Homestake Experiment detected neutrinos, some physicist were weary of the results. Conclusive evidence of the Solar Neutrino Problem was provided by the Kamiokande-II experiment, a water Cherenkov detector with a low enough energy threshold to detect neutrinos through neutrino-electron elastic scattering. In the elastic scattering interaction the electrons coming out of the point of reaction strongly point in the direction that the neutrino was traveling, away from the sun. While the neutrinos observed in Kamiokande-II were clearly from the sun, there was still a discrepancy between Kamiokande-II and Homestake; The Kamiokande-II experiment measured about 1/2 the predicted flux, rather than the 1/3 that the Homestake Experiment saw.

The solution to the solar neutrino problem was finally experimentally determined by the Sudbury Neutrino Observatory(SNO). The Ray Davis's Homestake Experiment was only sensitive to electron neutrinos, and the Kamiokande-II Experiment was dominated by the electron neutrino signal. The SNO experiment had the capability to see all three neutrino flavours. Because of this, it was possible to measure the electron neutrinos and total neutrino flux. The experiment demonstrated that the deficit was due to the MSW effect,  the conversion of electron neutrinos from their pure flavour state into the second neutrino mass eigenstate as they passed through a resonance due to the changing density of the sun. The resonance is energy dependent, and is visible near 2MeV. The water cherenkov detectors only detect neutrinos above about 5MeV, while the radiochemical experiments were sensitive to lower energy (0.8MeV for chlorine, 0.2MeV for gallium), and this turned out to be the source of the difference in the observed neutrino rates at the two types of experiments. Figure \ref{fig:solarexperiments} shows Homestake, Kamiokande-II and SNO experiments. 
\begin{figure}[htp]
\centering
	\begin{subfigure}[b]{.3\textwidth}
    \includegraphics[width=\textwidth]{figs/homestake.jpg}
    \caption{Ray Davis's Homestake Experiment}
    \label{fig:homestake}
    \end{subfigure}
    \quad
    \begin{subfigure}[b]{.3\textwidth}
    \includegraphics[width=\textwidth]{figs/kamiokande.jpg}
    \caption{Kamiokande Experiment}
    \label{fig:kamiokande}
    \end{subfigure}
    \quad
    \begin{subfigure}[b]{.3\textwidth}
    \includegraphics[width=\textwidth]{figs/sno.jpg}
    \caption{SNO Experiment}
    \label{fig:sno}
    \end{subfigure}
\caption{Solar Neutrino Experiments}
\label{fig:solarexperiments}
\end{figure}
\subsection{MSW Effect}
The Mikheyev-Smirnov-WOlfenshein effect is a process which acts to modify neutrino oscillations in matter. The presence of electrons in matter changes the energy levels of the mass eigenstates of neutrinos due to charged current coherent forward scattering of the electron neutrinos. This coherent forward scattering is similar to the electromagnetic process with respect to the refractive index of light in a medium. Because of this MSW Effect, neutrinos in vacuum have a different effective mass than neutrinos in matter and because neutrino oscillations depend on the squared mass difference of the neutrinos, the neutrino oscillations are different in matter than in vacuum. This effect is important at the sun where electron neutrinos are produced. The neutrinos of high energy leaving the sun are in a vacuum propagation eigenstate $\nu_{2}$ that has a very small overlap with the electron neutrino $\nu_{e}=\nu_{1}cos(\theta)+\nu_{2}sin(\theta)$ seen by the charged current reactions in Kamiokande-II and SNO. The discrepancy of the deficit between SNO, Kamiokande-II and Homestake is due to the energy of the solar neutrinos. The MSW effect "turns on" at about 2MeV and at lower energies, this MSW effect is negligible. \cite{Smirnov:2003da}



\section{Atmospheric Oscillations and the Atmospheric Neutrino Anomaly}
Atmospheric neutrinos are neutrinos that stem from the decay hadrons coming from primary cosmic rays. The dominant part of the decay chain is shown in equations  \ref{eq:piplus} and \ref{eq:piminus}

\begin{equation}
\label{eq:piplus}
\pi^{+} \rightarrow \mu^{+} \nu_{\mu} \mu^{+} \rightarrow e^{+} \nu_{e} \overline{\nu_{\mu}}
\end{equation}
\begin{equation}
\label{eq:piminus}
\pi^{-} \rightarrow \mu^{-} \overline{\nu_{\mu}} \mu^{-} \rightarrow e^{-} \overline{\nu_{e}} \nu_{\mu}
\end{equation}

\begin{figure}[htp]
\centering
\includegraphics[scale=.5]{figs/cosmicray.jpg}
\caption{Cosmic Ray Shower}
\label{fig:cosmicray}
\end{figure}

Figure \ref{fig:cosmicray} shows the cosmic ray shower. In general, these neutrinos have energies from 1GeV to 100s of GeV and the ratio of $\nu_{\mu}$s to $\nu_{e}$s equals to 2 (see equation \ref{eq:ratio})
\begin{equation}
\label{eq:ratio}
R = \frac{(\nu_{\mu} + \overline{\nu_{\mu}})}{(\nu_{e} + \overline{\nu_{e}})}
\end{equation} 



There have been two types of detectors used to study atmospheric neutrinos: Water Cherenkov detectors and tracking calorimeters. Super-Kamiokande is the detector we will focus on. These atmospheric detector experiments measure the ratio of $\nu_{\mu}$ to $\nu_{e}$. They also measure the zenith angle distribution of the neutrinos. These experiments report a double ratio (shown in equation \ref{eq:doubleratio}). This double ratio is the ratio measured in the detector to the ratio thats expected which is 2. If the double ratio equals to 1, the data agrees with the prediction. Various measurements from multiple experiments are shown in figure \ref{fig:ratiotable}. Except for Frejus, all R measurements are less than 1. This discrepancy between the predicted R and the measured R became known as the Atmospheric Neutrino Anomaly.
\begin{equation}
\label{eq:doubleratio}
R = \frac{(N_{\mu}/N_{e})_{DATA}}{(N_{\mu}/N_{e})_{SIM}}
\end{equation}

\begin{figure}[htp]
\centering
\includegraphics[scale=1]{figs/ratio.jpg}
\caption{Measurments of the double ratio for various atmospheric neutrino experiments}
\label{fig:ratiotable}
\end{figure}

Kamiokande-II has the the capability of measuring the direction of the incoming neutrinos. The expectation of atmospheric neutrino detection is that the flux be isotropic due to the fact that atmospheric neutrinos can reach the detector from all directions. Kamiokande-II noticed that muon-like data did not agree well with this expectation. At low energies approximately half of the $\nu_{\mu}$ are missing over the full range of zenith angles. At high energies the number of $\nu_{\mu}$ coming down from above the detector seems to agree with expectation, but half of the same $\nu_{\mu}$ coming up from below the detector are missing. This anomaly can be easily explained by neutrino flavour oscillations. Due to the fact that the neutrino travels less distance coming straight down into the detector (about 15km) than coming up from the bottom of the detector(13000km) changes the probability of oscillation. The probability of oscillation for the muon neutrinos coming down into the detector is roughly zero, whereas for neutrinos coming up, the oscillation probability is $sin^2(2\theta)$. Also, that fact that the electron-like events are not reduced, but the muon-like events are, suggests that the oscillation mode for atmospheric neutrinos is $\nu_{\mu} \rightarrow \nu_{\tau}$. 
Both the solar and atmospheric neutrino problems can be explained by neutrino oscillation so its fitting to derive this phenomenon mathematically. In the next two sections, two flavour and three flavour neutrino oscillation derivations will be explained. 


\section{Two Flavour Neutrino Oscillation Formulation}
The flavour eigenstates can oscillate between eachother because they are composed of an add-mixture of mass eigenstates($\nu_{1}$,$\nu_{2}$). Figure \ref{fig:mixing} shows the mass and flavour eigenstates rotated by an angle $\theta$ which is the mixing angle. 

In matrix form the wavefunctions are:

\begin{equation}
\begin{pmatrix}
\nu_{\mu} \\
\nu_{e}
\end{pmatrix} 
 = \begin{pmatrix}
cos\theta & sin\theta \\
-sin\theta & cos\theta 
\end{pmatrix}*
\begin{pmatrix}
\nu_{1} \\
\nu_{2} 
\end{pmatrix}
\end{equation}


\begin{figure}[htp]
\centering
\includegraphics[scale=.8]{figs/mixingangle.jpg}
\caption{The flavour eigenstates are rotated by an angle $\theta$ with respect to the mass eigenstates}
\label{fig:mixing}
\end{figure}

Applying the time evolution operator to $\nu_{\mu}$:
\begin{equation}
\centering
\label{eq:timeevol}
|\nu_{\mu}(t)> = -sin\theta|\nu_{1}>e^{-i\frac{E_{1}t}{\hbar}} + cos\theta|\nu_{2}>e^{-i\frac{E_{2}t}{\hbar}}
\end{equation}

where $E_{1} = \sqrt{p^2 c^2 + m^2_{1} c^4}$ and $E_{2} = \sqrt{p^2 c^2 + m^2_{2} c^4}$ and $p_{1}=p_{2}$. For the time being, let us assume $\hbar=c=1$. 
With this assumption:
$E_{1}=\sqrt{p^2+m^2_{1}}$ and $E_{2} = \sqrt{p^2 + m^2_{2}}$.
The next modifications is to assume neutrinos are relativistic:
\begin{equation}
\centering
\label{eq:gamma}
\gamma = \frac{E}{m^2_{o} c^2} = \frac{\sqrt{p^2 c^2 + m^2_{o} c^4}}{m_{o} c^2} \gg 1
\end{equation}
because of this,
\begin{equation}
\centering
p \gg m_{o}
\end{equation}
\begin{equation}
\centering
E = \sqrt{p^2 + m^2_{o}}=p\sqrt{1+m^2_{o}/p^2}\simeq p + \frac{1}{2}\frac{m^2_{o}}{p}
\end{equation}

where the binomial expansion is used. Now $E_{1}$ and $E_{2}$ can be written as:
\begin{equation}
\centering
E_{1} \simeq p + \frac{1}{2}\frac{m^2_{1}}{p} \mbox{    and    }  E_{2} \simeq p + \frac{1}{2}\frac{m^2_{2}}{p}
\end{equation} 
Now applying all these assumptions back into equation \ref{eq:timeevol} gives us:
\begin{equation}
\centering
|\nu_{\mu}(t)> = -sin\theta|\nu_{1}>e^{-i\left( p+\frac{1}{2}\frac{m^2_{1}}{p}\right)t} + cos\theta|\nu_{2}>e^{-i\left(p+\frac{1}{2}\frac{m^2_{2}}{p}\right)t}
\end{equation}
\begin{equation}
\centering
|\nu_{\mu}(t)> = e^{-i \left( p+\frac{1}{2}\frac{m^2_{1}-m^2_{2}}{p} \right) t} \left( -sin\theta|\nu_{1}> + cos\theta|\nu_{2}> \right)
\end{equation}
 Substituting $\Delta m^2 = m^2_{1}-m^2{2}$ and $t = \frac{x}{c} =x$ and $e^{-iz}= e^{-i \left( p+\frac{1}{2}\frac{m^2_{1}}{p} \right) t}$ gives us:
 \begin{equation}
 \centering
 |\nu_{\mu}(t)> = e^{-iz} \left( -sin\theta|\nu_{1}> + cos\theta|\nu_{2}>e^{+ix \left(\frac{1}{2}\frac{\Delta m^2}{p}\right)}\right)
 \end{equation}

Finding the Probability for a $\nu_{\mu} \rightarrow \nu_{e}$:
\begin{equation}
\centering
P(\nu_{\mu} \rightarrow \nu_{e}) = |<\nu_{e}|\nu_{\mu}(t)>|^2
\end{equation}

Remembering that $<\nu_{i}|\nu_{j}>=\delta_{ij}$
\begin{equation}
\centering
<\nu_{e}|\nu_{\mu}(t)> = e^{-iz}\left(-sin\theta cos\theta + sin\theta cos\theta e^{\frac{i \Delta m^2 x}{p}}\right)
\end{equation}
Taking the absolute value squared gives us:
\begin{equation}
\centering
P(\nu_{\mu} \rightarrow \nu_{e}) = |<\nu_{e}|\nu_{\mu}(t)>|^2 = e^{+iz}e^{-iz}sin^{2}\theta cos^{2}\theta\left( -1 + e^{\frac{i \Delta m^2 x}{p}}\right)\left(-1 + e^{\frac{i \Delta m^2 x}{p}}\right)
\end{equation}

Since the neutrino is relativistic we can set $p= E_{\nu}$ and change $x=L$. Also recognizing the trigonometric relation $(1 - cos2\theta)/2 = sin^{2}\theta$ the above equation becomes:
\begin{equation}
\centering
P(\nu_{\mu} \rightarrow \nu_{e}) = sin^{2}2\theta sin^2\left(\frac{\Delta m^2 L}{4 E_{\nu}}\right)
\end{equation}
 
 All that's left to do now is re-introduce $\hbar$ and $c$ doing this we get:
 \begin{equation}
 \centering
 P_{\nu_{\mu} \rightarrow \nu_{e}} (L,E) = sin^{2}2\theta sin^2\left(1.27 \Delta m^2 \frac{L}{E_{\nu}}\right)
 \end{equation}
 
 This equations has three important variables. 
 \begin{itemize}
 \item The angle $\theta$: This angle, as mentioned before, is called the mixing angle. It defines the difference between the flavour and the mass eigenstates. When $\theta = 0$ the mass and flavour eigenstates are identical and now oscillations occur. 
 \item The mass squared difference, $\Delta m^2$: Again $\Delta m^2 = m^2_{1}-m^2_{2}$. The reason this is an important variable is because it implies that for neutrinos to oscillate, neutrinos must have mass. Furthermore, the mass squared difference also tells us that the neutrino mass eigenstates must be different. 
 \item L/E: This is the variable that is of most interest to experimental physicists due to the fact that it is the variable that we set. L is the distance between the source and detector and E is the energy of the neutrino. For a given $\Delta m^2$, the probability of oscillation changes with respect to L/E. 
 \end{itemize}

\section{Three Flavour Neutrino Oscillation Formulation}
Seeing the quantum mechanics involved in deriving the probability of a two flavour neutrino oscillation, it is now possible to formulate the three flavour neutrino oscillation. The three flavour neutrino oscillation formulation begins  similarly to the two flavour, but there is the Pontecorvo-Maki-Nakagawa-Sakata matrix (PMNS) instead of the 2X2 matrix in the previous section. The PMNS matrix is show below:
\begin{equation}
\centering
\begin{pmatrix}
c_{12}c_{13} & s_{12}c_{13} & s_{13}e^{-i \delta} \\
-s_{12}c_{23} - c_{12}s_{23}s_{13}e^{i \delta} & c_{12}c_{23} -s_{12}s_{23}s_{13}e^{i \delta} & s_{23}c_{13}\\
s_{12}s_{23} - c_{12}c_{23}s_{13}e^{i \delta} & -c_{12}s_{23} - s_{12}c_{23}s_{13}e^{i \delta} & c_{23}c_{13}
\end{pmatrix} *
\begin{pmatrix}
e^{i \alpha_{1}/2} & 0 & 0\\
0 & e^{i \alpha_{2}/2} & 0\\
0 & 0 &1
\end{pmatrix}
\end{equation} 
where $c_{ij}=cos\theta_{ij}$ and $s_{ij}=sin\theta_{ij}$

Following the same steps as before we get:
\begin{equation}
\centering
P_{\alpha \rightarrow \beta}= \delta_{\alpha \beta} - 4 \Sigma Re(U^{*}_{\alpha i} U_{\beta i} U{\alpha j} U^{*}_{\beta j})sin^{2}\left(\frac{\Delta m^{2}_{ij} L}{4E}\right) 2 \Sigma Im(U^{*}_{\alpha i} U_{\beta i} U{\alpha j} U^{*}_{\beta j})sin\left(\frac{\Delta m^{2}_{ij} L}{2E}\right)
\end{equation}

The main things to notice here are $\delta_{ij}$ which is the CP violating term and has not been measured yet, and $\theta_{13}$ which has just been measured. CP violation is a violation of the postulated CP-symmetry. CP-symmetry states that the laws of physics should be the same if a particle were to be exchanged with its antiparticle and then if the left hand side of a decay were switched with the right hand side. 

%\section{Reactor Oscillation}
%Many experiments have searched for oscillation of electron anti-neutrinos produced at nuclear reactors. Such oscillations give the value of the parameter θ13. The KamLAND experiment, started in 2002, has made a high precision observation of reactor neutrino oscillation. Neutrinos produced in nuclear reactors have energies similar to solar neutrinos, a few MeV. The baselines of these experiments have ranged from tens of meters to over 100 km.
%On 8 March 2012, the Daya Bay team announced a 5.2σ discovery that θ13≠0. Two other experiments are currently measuring reactor neutrino oscillation (at the same baseline of a few kilometers) and may eventually confirm the Daya Bay results: Double Chooz and RENO.


%\section{Beam Oscillation}
%Neutrino beams produced at a particle accelerator offer the greatest control over the neutrinos being studied. Many experiments have taken place which study the same neutrino oscillations which take place in atmospheric neutrino oscillation, using neutrinos with a few GeV of energy and several hundred km baselines. The MINOS experiment recently announced that it observes consistency with the results of the K2K and Super-K experiments.
%The controversial observation of beam neutrino oscillation at the LSND experiment in 2006 was tested by MiniBooNE. Results from MiniBooNE appeared in Spring 2007, and appeared to contradict the findings of the LSND experiment. Results from the HARP-CDP group also put the LSND result into doubt.
%On 31 May 2010, the INFN and CERN announced having observed a tau particle in a muon neutrino beam in the OPERA detector located at Gran Sasso, 730 km away from the neutrino source in Geneva.
%The currently-running T2K experiment uses a neutrino beam directed through 295 km of earth, and will measure the parameter θ13. The experiment uses the Super-K detector. NOνA is a similar effort. This detector will use the same beam as MINOS and will have a baseline of 810 km.


%\section{Unanswered questions about Neutrinos }
%There are many questions that haven't been answered by previous experiments. Many of them I have touched upon in previous sections, but here is a formal list of questions left unanswered. 
%\begin{itemize}
%\item What is the sign of $\Delta m^{2}_{23}$?
%\item Are there more than three active neutrinos? (LSND experiment)
%\item Are the neutrinos and antineutrinos identical?
%\item Is there a CP violating term that needs to addressed. 
%\item Will $\delta_{cp}$ explain matter-antimatter asymmetry?  
%\item Why is the PMNS matrix so different in form to the CKM matrix( quark mixing matrix)
%\end{itemize}


%\section{Final Thoughts}
%Although there is much about neutrinos that we don't know, we have made much progress in trying to understand these elusive particles. We now know that neutrinos aren't massless and because of this knowledge we now know that neutrinos oscillate between flavour and mass eigenstates. We have also just recently discovered that $\theta_{13}$ is non-zero and this knowledge will lead to experiments and data analysis to find $\delta_{cp}$.
%\clearpage

%More introduction

  \chapter{The MicroBooNE Experiment}\label{ch:microboone}
The purpose of this chapter is to discuss and understand the details of the MicroBooNE detector. A thorough understanding of MicroBooNE and the technology behind liquid argon time projection chambers is important for understanding results as well as understanding how images were made for use in deep learning efforts that will be outlined in later chapters.   

\section{Liquid argon time projection chambers}
Liquid Argon Time Projection Chambers (LArTPCs) are an exciting detector technology that provide excellent imaging and particle identification, and are now being used to study neutrinos. The Time Projection Chamber (TPC) was first invented by Nygren in 1974 \cite{nygren} and the proposal for a LArTPC for neutrino physics was made by Rubbia \cite{rubbia} in 1977 with the ICARUS collaboration implementing this concept\cite{icaraus}. A LArTPC is a three-dimensional imaging detector that uses planes of wires at the edge of an active volume to read out an interaction. When a neutrino interacts with an argon atom, the charged particles that are produced ionize the LAr as they travel away from the interaction. By placing a uniform electric field throughout the LAr volume, the ionization is made to drift towards a set of anode planes, which consist of wires spaced very closely together collecting the ionized charge, which is subsequently read out by electronics connected to the anode wires. The collected ionization creates a spatial image of what happened in the detector on each anode plane. The position resolution of the interaction along the beam direction (perpendicular to drift direction) relies on the wire pitch, while the resolution in drift direction is dependent on the timing resolution of the electronics used and the longitudinal diffusion in the volume. The drift time of the ionization relative to the time of the original signal allows the signal to be projected back along the drift coordinate, hence the name LArTPC. Having very small distances between each wire within an anode plane allows for very fine granularity and detail to be captured, and having multiple wire planes at different angles provides independent two-dimensional views that can be combined into a three dimensional picture of the interaction. Once the charge signal is created on the anode planes, software analysis packages identify particles in the detector by using deposited energy on the wires along their track length. 
The 30 year development of the ICARUS detector has led to LArTPCs being used as cosmic ray \cite{lartpc_cosmic}, solar neutrino \cite{lartpc_solar} and accelerator neutrino \cite{lartpc_accelerator} detectors. The ArgoNeuT experiment at Fermilab was the first United States based liquid argon neutrino program that has since produced short-baseline $\nu-Ar$ cross-section measurements in the NUMI beamline \cite{argoneut}. The MicroBooNE experiment is the second experiment in the US based LArTPC neutrino program and will be discussed thoroughly in the next sections.  
The next phases of the liquid argon neutrino program are under way and are the Fermilab Short Baseline Neutrino (SBN) program \cite{sbn} and the Deep Underground Neutrino Experiment (DUNE) \cite{dune}. The SBN program will include three LArTPC detectors, including the MicroBooNE detector, on the Booster Neutrino Beam (BNB) to do multiple-baseline oscillation measurements. The detector closest to the beam will be the 40 ton Short Baseline Neutrino Detector (SBND)\cite{sbnd} at 150 m and the detector furthest is the 600 ton ICARUS T600 \cite{icarus_t600} detector positioned at 600 m. The DUNE collaboration will deliver a 30 GeV neutrino beam 1300 km from Fermilab to a 34 kiloton LArTPC detector at Homestake, SD. DUNE will study the leptonic CP phase, $\delta_{cp}$, as well as measure neutrino and antineutrino oscillations. 
\section{The MicroBooNE Time Projection Chamber}The $\nu-Ar$ interaction also produces scintillation light which is collected by photomultiplier tubes (PMTs) which allows the exact time of the neutrino interaction to be determined.
MicroBooNE (Micro Booster Neutrino Experiment) is a ~89 T active volume (180 T total mass) LArTPC which is then inserted into a cylindrical crysotat on axis of the Booster Neutrino Beam (BNB) stationed at Fermilab in Batavia, Illinois. The main components of MicroBooNE will be detailed in the upcoming sections. MicroBooNE is also an R\&D detector that can be scaled up to a significanlty larger size, such as Deep Underground Neutrino Experiment (DUNE) which is roughly 40 kT compared to MicroBooNE at 180T \cite{dune}. Understanding LArTPC technology and detector physics is necessary to build a LArTPC the size of DUNE, and MicroBooNE has made many advances in developing this technology\cite{noisechar} \cite{michel}. 

MicroBooNE's Time Projection Chamber (TPC) is 10.3 m long (beamline direction), 2.3 m high and 2.5 m wide (which corresponds to the drift distance). The TPC is shown in figure \ref{fig:tpc}. MicroBooNE is the largest LArTPC currently running in the world\cite{microboone}. This LArTPC has 3 wire planes: 1 plane that collects the ionization in the wires and is $0^{\circ}$ to the virtical with 3456 wires spaced 3 mm apart, and 2 planes where the ionization drifts passed and induces a signal at $\pm 60^{\circ}$ to the vertical each with 2400 wires also spaced 3 mm apart. Each plane has a spacing also of 3 mm from eachother. The wires are then connected to detector specific circuit boards (ASICS) that are submered and operate inside liquid argon. The first two planes are the induction planes and the last is the collection. The electric field of the TPC is created using 64 stainless steel tubes shaped into rectangules around the TPC and held in place by G10 to form a field cage. The cathode is charged at a high voltage of -70 kV and this voltage is stepped down across the field cage tubes using a voltage divider chain with an equivalent resitance of 240 $M\ohm$ between the tubes. The field cage tubes are separated by 4 cm from center to center. 
\section{Light Collection System}
The light collection system is a crucial part for 3D reconstruction of particle interactions in the LArTPC. It is possible to reconstruct interactions using just the wire signals, but without the initial timing (t0) of an event, it is impossible to position the event along the drift direction. When a particle interaction occurs, the scintillation light created propogates within nanoseconds to the light collection system compared to the milliseconds it takes the ionized electrons from the interaction to reach the anode wire planes. Therefore we can precisely know where along the drift direction the particle interaction first took place. The scintillation light is also localized, so combining the PMT information with the wire plane information allows for cosmic background rejection happening outside the beam timing window.  

The light collection system is made up of 32 Hammamatsu R5912-02mod cryogenic PMTs with a diameter of 8-inches. The PMTs are are located behind the 3 wire anode planes and provides 0.85\% photocathode coverage. Each PMT has an acrylic plate mounted in front of it that is coated with a wave-length shifting material called TPB. The acrylic plates take in the scintillation light, at 128 nm, and re-emits it visible wavelengths visible to the PMTs, with a peak at 425 nm. 
\section{Electronics System} 
More MicroBooNE stuff. Possibly talk about rack protection system work I did i.e circuit board soudering? Talk about deconvolution paper Adam and I wrote a while back?

  %\chapter{The Booster Neutrino Beam}
Chapter about the beam
\clearpage
\dots

  \chapter{Neutrino Identification: Finding MicroBooNE's first Neutrinos} \label{ch:neutrinoID}
The goal of the Neutrino Identification analysis was to positively identify BNB neutrino interactions in the MicroBooNE detector collected during the first days of running. Neutrino event candidates were identified in part by using a cut on detected flash of scintillation light during the 1.6 $\micro s$ beam-spill length of the BNB as well as identifying reconstructed object from the TPC that are neutrino like. After this selection, 2D and 3D event displays were used for verification of the selection performance. This selection was targeted to reduce the ratio of neutrino events to cosmic-only events from the initial 1 neutrino to 675 cosmics to a ratio of 1 to 0.5 or better which is equivalent to a background reduction by a factor of 1000 or more. These selected events were used for MicroBooNE's public displays of neutrino interactions. A clearly visible neutrino interaction with an identifiable vertex and at least 2 tracks originating from the vertex was what the analysis focused on. This analysis wasn't optimized for high purity or efficiency, but rather for very distinguishable neutrino interactions that could be identified by the public.
\section{Flash Finding}\label{sec:flashfinding}
Flash finding is the first step used in finding neutrino interactions. This section will detail how optical information is reconstructed as well as analysis scripts and event filters were used.
\subsection{Flash Reconstruction}
\begin{figure}[htp!]
\centering
	\begin{subfigure}[b]{.6\textwidth}
	\includegraphics[width=\textwidth]{figs/totalpecut.png}
	\end{subfigure}
	\quad
	\begin{subfigure}[b]{.6\textwidth}
	\includegraphics[width=\textwidth]{figs/totalpe_zoomed.png}
	\end{subfigure}
	\quad
\caption{Efficiency for selecting beam events as a function of minimum total PE cut for all PE cuts as well as zoomed into interesting region.}
\label{fig:PE}
\end{figure}
A flash is described as a collection of light seen at the same time within the detector. They are then reconstructed by identifying signal from the PMTs above a specific photoelectron (PE) threshold. These signals are called optical hits. Optical hits from all the PMTs are then accumulated into 1 $\micro s$ bins of time. If a specific bin is above a set PE threshold, then the optical hits that overlap in time are the labeled as the hits from the flash. All flash reconstructed properties like average time and x/y positions are then found via the flash labeled optical hits. The total size of the flash is found by summing up the total number of photoelectrons from all PMTs. Neutrino interactions and cosmic muons will have a larger flash size compared to noise and other low-energy backgrounds, therefore a total PE cut is used to reject these backgrounds. A total PE cut of 50 PE was deemed sufficient for this analysis. Figure \ref{fig:PE} show the total PE versus the selection efficency of selecting neutrino beam events. 


\subsection{Beam Timing}
\begin{figure}[htp!]
\centering
	\begin{subfigure}[b]{.6\textwidth}
	\includegraphics[width=\textwidth]{figs/flashrate_sim.png}
	\caption{Predicted distribution of flash times with respect to trigger time for 1 day of data taking at nominal rate and intensity}
	\label{fig:pe_sim}
	\end{subfigure}
	\quad
	\begin{subfigure}[b]{.6\textwidth}
	\includegraphics[width=\textwidth]{figs/flashrate.png}
	\caption{Measured distribution of flash times with a 50 PE threshold cut, with respect to trigger time. Shown as a ratio to the expected cosmic rate from off-beam data. A clear excess from neutrinos is visible between 3- 5 $\micro s$ after the trigger time. }
	\label{fig:pe_data}
	\end{subfigure}
	\quad
\label{fig:petime}
\end{figure}
It is necessary to get the specific time from flashes if one uses flashes to filter out neutrino interactions coincident with the neutrino beam spill period and background. Before a filter can be applied, an understanding of the timing of the trigger and PMT readout with respect to the arrival of neutrinos from the BNB. To do this, a 1.6 $\micro s$ window near the expected beamtime was created and verified by finding that the number of flashes was significantly above the cosmic-ray background flashes. Beam data during the first week of running, October 16th 2016 through October 22nd 2016 and were used for a timing measurement. The total POT uses corresponds to roughly 24 hours of data taking at nominal intensity ($4 X 10^{12} ppp$) and a 5 Hz repetition rate. Figure \ref{fig:pe_sim} shows size of the expected neutrino signal in time using Monte Carlo predictions and figure \ref{fig:pe_data} shows the neutrino signal in data. The intensity in data is lower, however there can still be seen a significan excess above data.

\subsection{Event Rates}
Applying a 50 PE threshold cut inside a 1.6 $\micro s$ window reduces the cosmic-ray passing rate to 0.8\%. With a 5 Hz beam rate, this corresponds to 135 cosmics passing per hour. The neutrino passing rate for this filter is about 22 events per hour. To further increase the neutrino to cosmic ratio, TPC topology cuts were implemented and will be discussed in the following section.
\section{TPC Topology Selection}  
In order to further reduce the background of cosmic events, two independent selection streams using TPC wire data reconstruction was implemented. The first using 2D reconstructed clusters, and the second using 3D resonctructed tracks. Both streams look for neutrino interactions in the active TPC volume which are identifiable by two or more tracks originating from the same vertex.

Both 2D and 3D channels were optimized using monte carlo simulation which used a 128 kV cathode voltage. Passing rates were calculated using a 0.008 efficiency factor for cosmic events passing to simulate the flash finding described in section \ref{sec:flashfinding}. This efficiency factor was an overestimation and was just used to get a general feel of what signal and background rates we would actually see in data. 
\subsection{Cosmic Tagging}
The first step in TPC selection was based on the geometry of cosmic tracks in an event. The cosmic ray muon geometry tagger runs on 3D tracks and assigns a score to each reconstructed track on the likeliness of the track originating from a cosmic. The cosmic scores are detailed below:
\begin{itemize}
\item 1: The track is tagged as entering or entering the TPC
\item 0.95: The track is a delta ray associated with a tagged track
\item 0.5: The track is either entering or exiting, but not both
\item 0.4: The track is entering or exiting through the Z boundary
\item 0: The track isn't tagged
\end{itemize}
Clusters are assigned either a 0 or 1, 1 being a cosmic. In simulation, 90\% of cosmics are tagged as cosmics. These tracks are no longer considered when looking for a neutrino topology. Requiring that the tracks be contained in turn affects the neutrino efficiency by 20\%. The algorithm checks that each track is contained within a boundary region of 10 cm from all sides of the TPC. This boundary region was optimized via handscanning of experimental data.

As can be expected, cosmic tagging is more efficient in the 3D channel (tracks) than the 2D channel (clusters) because the reconstructed tracks can use the full 3D position information of the entering and exiting points while the 2D channel mainly use the reconstructed x position of the cluster which is associated to timing. 

Cosmic tagging uses timing information to reject tracks and clusters that are outside of drift window. The drift window for 128 kV is 1.6 $\micro s$ while for 70 kv, the actual voltage MicroBooNE is running at, is 2.3 $\micro s$. Due to this variation between simulation and data, we expect to see $2.3/1.6 = 1.44$ times more cosmic induced tracks or clusters in the drift window. 
\subsection{2D Cluster Selection}
This selection was spearheaded by myself and Katherine Woodruff. After looking at experimental cosmics data, 2D clustering performs well, while 3D track reconstruction is affected by more variations in simulation, for example noise filters. This was the motivation for having a selction only on 2D clusters in the collection (Y) plane. As stated previously, the goal of this analysis was to find identifiable neutrino interactions for use in public event displays, in future analyses, the 3D track reconstruction has been modified to further increase the tracking efficiency and has more information that just the clusters. For this analysis, however, 2D cluster information was sufficient enough for neutrino selection. 
\subsubsection{Primary Cuts}
The first cuts were used to select which clusters to consider. First the clusters must have at least ten hits on the collection plane and have a cosmic tagging score < 0.4. Only events that have at least two clusters that satisfy these primary cuts continue on.

After the initial cosmic tagging is applied, the following cuts are used to furture separate identifiable neutrinos for background cosmics. 

The next cut was to remove long, vertical clusters. This was applied after seeing that most cosmic induced clusters passing were long with high angles, while neutrino induced clusters were mainly forward going. We required a good cluster to either have a projected start angle less than 30 degrees from the z axis or be less than 200 wires long. The length cut was added to make sure we don't cut any short high angle clusters that can correspond with a proton, or other highly ionizing particle associated with a long muon cluster. The 200 wire cut roughly equates to 0.6 m in the z direction, with a 3 mm wire pitch. Also, the projected angle is defined by tan $\alpha$ = $\Delta T / \Delta W$ where T is the time ticks and W is the wires. 

The last cut requires the clusters to be either 30 time ticks or 30 wires. This cut was applied to reduce small delta rays associated with a cosmic without removing proton clusters associated with a long muon cluster, which saves ideal neutrino events that have both a long minimum ionizing muon like cluster and a short highly ionizing proton like cluster.

\subsubsection{Secondary Cuts}
The secondary cuts look to match long, low-angle clusters with short, high-charge clusters. Only clusters that have passed previous cuts are used. First clusters with length greater than 100 wires are chosen, which is approximately 0.3 m in the z direction. Then we search for any cluster that is within approximately 3 cm ( 10 wires and 30 time ticks) away from the low-z end of the long cluster. This cluster must also be shorter than the first. In our reconstruction, the start and end point of a cluster can be swapped so both ends of the short cluster are compared to the long cluster. 

Now that there is a vertex match, cuts based on charge and projected opening angle are implemented. We require the short cluster to have a higher start charge than the long cluster or the long cluster be longer than 500 wires. Start charge is defined as the charge on the first wire in ADC counts. The projected opening angle must also be between 11 and 90 degrees. This last cut is intended to remove clusters that are entirely overlapping or are part of the same long track. 
\begin{table}
\begin{center}
\begin{tabular}{c c c c}
\hline
Cluster set & No Cuts & Primary Cuts & Secondary Cuts \\
\hline
Neutrinos only & 570 & 303 & 32 \\
Cosmics only ( no flash) & 308,016 & 291,879 & 602 \\
Cosmics only (w/ flash) & 2464 & 2335 & 5 \\
\hline
Neutrinos/Cosmics & 0.23 & 0.13 & 6.4 \\
\end{tabular}
\caption{Passing rates for 2D cluster cuts for neutrino on MC set and a cosmic only MC set. First column shows event rates with no cuts applied to both sets. Columns two and three show event rates after primary and secondary cuts are applied. Line three shows the seccond line scaled with the flash finding factor of 0.008. All events are normalized to per day assuming we are running at 5 Hz.}
\label{table:eventrate}
\end{center}
\end{table}
The resulting neutrino/cosmic event rate per day is shown in table \ref{table:eventrate}. Figures \ref{fig:2dprimarycut} and \ref{fig:2dsecondarycuts} shows the percentages of clusters that pass each primary and secondary cuts.     

\begin{figure}[htp!]
\centering
\includegraphics[width=\textwidth]{figs/2dprimarycut.png}
\caption{Percent of good clusters remaining for neutrinos and cosmics after the primary cuts were applied. This is relative to total number of initial clusters.}
\label{fig:2dprimarycut}
\end{figure}

\begin{figure}[htp!]
\centering
\includegraphics[width=\textwidth]{figs/2dsecondarycut.png}
\caption{Percent matched cluster pairs remaining for neutrinos and cosmics after secondary cuts applied. This is relative to the number of events that contain clusters which pass the primary cuts.}
\label{fig:2dsecondarycuts}
\end{figure}

\subsection{3D Tracks and vertices Selection}
The neutrino selection for the 3D channel was based on a reconstructed vertex and two tracks. All vertices and tracks were looped over that had a cosmic tag score < 0.4 and the distances below were calculated:
\begin{itemize}
\item d: distance between the start points of the two tracks.
\item $d_1$: distance between vertex and start of track 1.
\item $d_2$: distance between vertex and start of track 2.
\end{itemize}
The maximum distance of all three is then selected as the important characteristic per trio. The best trio is the one that has the smallest maximum distance. The $min(max_{d})$ for all trios in an event were plotted for BNB neutrino events and for cosmics to find the best cut value for each tracking algorithm. The distribution of $min(max_{d,i})$ is smaller for neutrinos than for cosmics. The cut values for different tracking and clustering algorithms are shown below. These cut values were chosen to minimize the cosmic background to 20\%. 
\begin{itemize}
\item trackkalmanhit with cccluster $min(max_{d,i})$ < 3 cm.
\item trackkalmanhit with pandoraNu $min(max_{d,i})$ < 4.5 cm.
\item pandoraNu with cccluster $min(max_{d,i})$ < 5 cm.
\end{itemize}

\subsection{TPC Updates}
After doing a visual hand-scanning of the first beam data processed with the filters detailed above, the events passing had a larger contamination of background than expected. This was mainly in part due to the reconstruction performing better on simulation than on data. Due to this, additional cuts on both streams needed to be implemented in order to increase signal/background ratio. These cuts were added on top of the filters described above and further reduce the event count. 
\subsubsection{2D Filter Updates}
The main background observed in the 2D filter were Michel events, where the muon and electron formed two connected clusters. These events were rejected by comparing the start and end charge deposition of the long cluster (i.e muon particle). The start charge deposition must be less than the end charge deposition. This cut is implemented because muons have a higher ionizaiton loss at the end. 
\subsubsection{3D Filter Updates}
It was seen that cosmic tracks can often originate or end at the same point, therefore faking a signal. Cosmic tracks, however, are mostly vertical. By requiring the angle of the longer track have a cosine greater than 0.85 with respect to the z-axis as well as requiring the longer track to have a length greater than 10 cm, we can reduce this background. 
\section{Conclusion}
After proccesing these filters in parallel, it was shown that the 3D filter had a higher purity than the 2D filter because of the higher cosmic rejection being used due to 3D reconstruction. The 2D filter is blind to track entering/exiting from the top or bottom of the TPC. Although the 3D filter had a higher purity, the 2D filter was still able to find identifiable events in data that were used as public event displays. A sample of event displays are shown in figures \ref{fig:2dimage} and \ref{fig:3dimage}.

  \chapter{CC-Inclusive Cross Section Selection Filter} \label{ch:meas}
The CC-Inclusive cross-section selection I and selection I modified filters used in this analysis will be described in the following sections below. These filters are an expansion of the Neutrino ID filter. The work done in this thesis was to further improve these selections by increasing both efficiency and purity without further affecting the kinematic distributions of the selected neutrino events.
 
MicroBooNE requires fully automated event reconstruction and selection algorithms for use in the many physics measurements being worked on to date due to the large data rate MicroBooNE receives. Being able to automatically pluck out the neutrino interaction among a sea of cosmics proved to be challenging but was accomplished. MicroBooNE has developed two complementary and preliminary selection algorithms to select charged-current $\nu_{\mu}-Ar$ interactions. Both are fully automated and cut based. The results below focus on the first selection and the ``In-Progress'' plots presented on the poster associated with this proceeding will focus on further improving this algorithm using Convolutional Neural Network (CNN) implementations. The full details can be found in MicroBooNE public note \cite{ccinclusive} and for more information of CNN implementation on MicroBooNE data refer to \cite{cnn}. Selection I is based on cuts developed in a MC performance study described in \cite{mcccinclusive}. It identifies the muon from a neutrino interaction without biasing towards track multiplicity. To combat cosmic and neutral current background, the analysis is strongly biased towards forward-going long tracks which are contained. This limits phase space and reduces acceptance. 

The efficiency and purity are used as performance values of selection I. Efficiency is described as the number of selected true $\nu_{\mu}$ CC events divided by the number of expected true $\nu_{\mu}$ CC events. The purity is described as the number of selected true $\nu_{\mu}$ CC events divided by the sum of itself and all the backgrounds. The efficiency of selection I is 12\% and the purity is 39.7\%. The poster related to this proceedings will focus on the last cut which requires the longest track to be longer than 75 cm. This cut has a passing rate of 30\% w.r.t the previous cut and is implemented in part to separate charged-current events from neutral-current events that mimic our signal. Implementing a CNN for $\mu-\pi$ separation picks out differences in these two particles that are track range independent therefore eliminating the need for the 75 cm track length cut and increase efficiency and passing rate at low muon momentum. Figure \ref{fig:track} shows the track distribution of selection I and the lack of data below  the 75 cm track length cut. Figure \ref{fig:eff} shows the efficiency of selection I as a function of muon momentum.     
The selection begins with a cut that requires an optical flash greater than 50 photo electrons (PE) in the 1.6 $\mu$s beam window. Next, two or more 3D reconstructed tracks must be within 5 cm from a 3D reconstructed vertex. The most forward going track vertex-track association is then selected for further cuts. The vertex from the chosen association must be in the fiducial volume, and the longest track from this association must be matched to a flash 80 cm in z. Lastly the longest track must be contained and longer than 75 cm.       

\begin{figure}[htp!]
\centering
	\begin{subfigure}[b]{.4\textwidth}
	\includegraphics[width=\textwidth]{figs/track_distribution.png}
	\caption{Track range distribution of selection I}
	\label{fig:track}
	\end{subfigure}
	\quad	
	\begin{subfigure}[b]{.4\textwidth}
	\includegraphics[width=\textwidth]{figs/efficiencyvsmom.png}
	\caption{Selection efficiency as a function of the true muon momentum}
	\label{fig:eff}
	\end{subfigure}
	\quad
\label{fig:distributions}
\caption{\ref{fig:track} Track range distribution for selection I. The track range is defined as the 3D distance between the start and end of the muon candidate track. No data is shown below 75 cm due to the track length cut described previously. \ref{fig:eff} Efficiency of the selected events by process quasi-elastic (QE), resonant (RES), and deep-inelastic (DIS). Statistical uncertainty is shown in the bands and the distributions are a function of true muon momentum. The rise of the efficiency between 0 GeV and 0.5 GeV is due to the minimum track length cut and the decreasing efficiency for higher momentum tracks is caused by the containment requirement.} 
\end{figure}

\section{The importance of $\mu/\pi$ separation}

  %\chapter{The importance of $\mu/\pi$ separation in MicroBooNE}
$\mu/\pi$ separation chapter goes here.

\clearpage

More $\mu/\pi$ separation.

  \chapter{Background on Convolutional Neural Networks}\label{ch:cnn}
Convolutional neural networks (CNNs) have been one of the most influential innovations in the field of computer vision. Neural networks became popular in 2012 when Alex Krizhevsky used them to win that year's ImageNet competition\cite{aleximagenet} by dropping the error from 26\% to 15\%. Since then, many companies are using deep learning including Facebook's tagging algorithms, Google for their photo search and Amazon for product recommendations. For the purpose of this thesis CNNs were used for image classification, specifically, images of varying particles created using LArTPC data. 

\section{Image Classification} 
Image classification is the process of inputting an image into the CNN an receiving a probability of classes that best describes what is happening in the image. As humans, image classification is something that is learned at a very young age and is easy to do without much effort. This is also apparent when hand-scanning LArTPC images. After learning what a neutrino event looks like in MicroBooNE, it is relatively easy to recognize simple neutrino events from cosmic ray background as well as highly ionizing particles like protons from minimum ionizing particles like muons. The very detailed images LArTPC detectors output are prime candidates for input images into a CNN. CNNs mimic a human's ability to classify objects by creating an architecture that can learn differences between all the images it's given as well as figure out the unique features that make up each object. CNNs are modeled after the visual cortex. Hubel and Wiesel\cite{hubel} found that there are small regions of neuronal cells in the brain that respond to specific regions of the visual field. They saw that some neurons fired when exposed to vertical edges while others fired when shown horizontal or diagonal edges. They also saw that these neurons were organized in columns. The idea of specific neurons inside of the brain firing to specific characteristics is the basis behind CNNs.

\subsection{CNN Structure}
When used for image recognition, convolutional neural networks consist of multiple layers that extract different information on small portions of the input image. How many layers is tunable to increase the accuracy. The output of these collections are then tiled so that they overlap to gain a better representation of the original image and allow for translation. The first of these layers is always a convolution layer. To the CNN, an image is an array of pixel values. For a RGB color image with width and height equal to 32x32 the corresponding array is 32x32x3. Filters, also known as neurons, of any size set by the user is then convolved with the receptive field of the image. If the filter is 5x5, the receptive field will by a patch of 5x5 on the input image. The filter is also an array of numbers called weights. The convolution of the filter and image are matrix multiplications of the weights and the pixel values. By stepping the receptive field by 1 unit, for an input image of 32x32x3 and a filter of 5x5x3 you'd get an output array of 28x28x1. This output array is called an activation map or feature map. The use of more filters preserves the spatial dimensions better. The filters can be described as feature identifiers. Examples of features in an image consist of edges, curves, and changes in colors. The first filters in a CNN will primarily be straight line and curve feature identifiers. An example of a curve filter is shown in figure \ref{fig:curvedetector}. When a curve in the same concavity is found in the input image, the corresponding pixel in the output feature map will be activated. Going back to our example of a 32x32 input image and a 5x5 filter, if there were to be a curve in the top left corner of the input image, our output feature map would have a high pixel value in the top left. Therefore, feature maps tell us where a specific feature is located in the original image. Figure \ref{fig:conv1} shows a visualization of filters found in the first layers of many CNN architectures. These filters in the first layer convlove around the image and activate whe nthe specific feature it is looking for is in the receptive field. 
\begin{figure}[htp!]
\centering
\includegraphics[width=.6\textwidth]{figs/curvedetector.png}
\caption{Pixel representation and visualization of a curve detector filter. As you can see, in the pixel representation, the weights of this filter are greater along a curve we are trying to find in the input image}
\label{fig:curvedetector}
\end{figure} 

\begin{figure}[htp!]
\centering
\includegraphics[width=.25\textwidth]{figs/conv1vis.png}
\caption{Visualization of filters found in first layer of a CNN.}
\label{fig:curvedetector}
\end{figure} 

In figure \ref{fig:convolution} you can see how an edge detection filter is used to save only necessary information for recognizing different types of clothes. You can also see by having multiple filters you can get more detail or less detail from an image which can then simplify or complicate the object recognition task. Being able to distinguish between a shirt or a leg garment is as much information you want, having a filter that extracts outline edge or shape information would be all that you need. But if instead you wanted to distinguish between a formal cocktail dress or a summer dress, more information would need to be saved equating to many more filters for one image. Rather than trying to come up with how many filters and what features are important for detection, CNNs do this automatically. CNNs take input parameters, called hyper-parameters, for example number of layers, number of filters per layers, number of weights per filter, and uses these to create the output feature maps. The layers build upon each-other, for example if we were creating a CNN for facial recognition the convolutional layers will start learning feature combinations off of the previous layers. The low level features like edges, gradients, and corners of the first layers become high level features like eyes, noses, and hairs. This process is visualized in figure \ref{fig:featuremaps} 

\begin{figure}[t!]
\centering
\includegraphics[width=.48\linewidth]{figs/convolution.png}
\caption{Applying a feature mask over a set of fashion items to extract necessary information for auto-encoding. Unnecessary information for example color or brand emblems are not saved. This feature map is an edge detection mask that leaves only shape information which helps to distinguish between different types of clothes.} 
\label{fig:convolution}
\end{figure}

\begin{figure}[h!]
\centering
\includegraphics[width=.9\linewidth]{figs/facialDetection.png}
\caption{Pictorial Representation of Convolutional Neural Networks as well as a visual representation on CNN's complexity of layer feature extraction}
\label{fig:featuremaps}
\end{figure}
There are other layers in a CNN architecture that will not be covered in the scope of this thesis but in a general sense, these layers are interspersed between convolution layers to preserve dimensionality and control overfitting of the network. The last layer is called a fully connected layer and it's job is to output an N dimensional vector where N is the number of classes the network has been trained on. Each number in this vector represents the probability that the input image is a certain class. Fully connected layers use the feature maps of the high level features to compute the products between the weights of the previous layer to get the probabilites of each class. These weights are then adjusted throught the training process using backpropagation. 
\subsection{Backpropagation}

\section{AlexNet}
\section{GoogleNet}

  %\chapter{Hardware Frameworks}\label{ch:hardware}
Chapter about different frameworks used/timing on each setup.
\section{Syracuse CPU Machine setup}
\section{Syracuse University GPU Cluster Setup}
\dots

  \chapter{Training Convolutional Neural Networks on particles \textcolor{red}{\textbf{WORKING TITLE}}}\label{ch:cnn_train}
Three Convolutional Neural Networks CNNs were trained throughout this analysis. There are differences to each CNN and will be described fully in the next sections but the main difference are the amount of particle images used for training and validation. CNN1075 used 1,075 muons and 10,75 pions for training and the same amout of each particle for validation. CNN10000 used 10,000 muons and 10,000 pions split in half for testing and training. Lastly CNN100000 had muons, pions, protons, electrons, and gammas in it's training and validation set. Each particle had 20,000 images and training and validation was split $90\%$ training, $10\%$ validation. This chapter will also describe the different hardware frameworks used for training beginning on a CPU and ending on a GPU cluster.  


\section{Hardware Frameworks used for Training}
\subsection{Syracuse CPU Machine setup}
\subsection{Syracuse University GPU Cluster Setup}


\section{Convolutional Neural Network Training}\label{research approach}

\subsection{Image Making Scheme}\label{image_making} 
\subsubsection{Images used for Traing/Validation of Convolutional Neural Networks}

\textcolor{red}{\textbf{add image making for CNN1075}}
The $\mu/\pi$ image dataset used to train and validate the CNN10000 was created using single generated isotropic muons and pions from 0-2 GeV energ range. 10,000 muons and 10,000 pions were used for training and testing split 50\%. The images were created based on wire number and time tick in the collection plane. Uboonecode v06{\_}23{\_}00 was used instead of v05{\_}08{\_}00 which was used previously. The wire signal was the raw ADC value after noise filtering. Each collection plane grayscale image was 3456x1280x1 where 5 time ticks were pooled into 1 bin which is different than the previous dataset and was implemented due to the fact that the time ticks of an event went from 9400 to 6400 with the change of uboonecode version. The grayscale color standard is 8bit therefore the ADC value of wire and time tick was also downsampled due to the 12bit ADC value MicroBooNE has. To do this, the highest ADC pixel in the image was found and then this was divided by the rest placing all pixel values between 0-1. From there, all pixel values are then multiplied by 255. All images were made using a LArSoft module. Once the images were created, using and image manipulation framework called OpenCV images were read into a numpy array and cropped to the region of interest by only keeping rows and columns where all ADC values are higher than 0 and then resized it to 224x224 using OpenCV's resize function. This downsampling of ADC values creates a problem of information loss for example, a proton which is highly ionizing will have the same brightness as a minimum ionizing muon by virtue of how the images are created. 
Issues that arose in CNN1075 that were fixed in CNN10000 include zero-padding images in X and Y that are smaller than 224X224 to eliminate over-zooming effect and fixing a bug that shifted pixels separated by a dead-wire region. 

Images were also made from events that passed the cc-inclusive selection 1 filter right before the 75 cm track length cut and were classified using the CNN10000. The dataset used to create these images is the same one used in \cite{cc-inclusive}, prodgenie{\_}bnb{\_}nu{\_}cosmic{\_}uboone{\_}mcc7{\_}reco2. These images were created using information from the track candidate that passed the filter. Only wire number and time ticks associated to the track candidate were drawn on the image to mimic a single particle generated image. These images were then classified using CNN10000. Two approaches were taken in making these images. The first was using the image normalization above where the maximum pixel in each image is used as a normalization constant to get all pixels between 0-1 then multiply all pixels by 255. As described above, this is the incorrect way to normalize; it should be normalized by dataset not by event, which is the second way the images were created. The results of CNN10000 performance are shown in section \ref{research approach}. 

\subsection{Training CNN1075}
The work shown in these next sections are based on the previous work done described in \cite{priorwork}. That CNN (now referred to as CNN1075) was trained using single generated isotropic muons and pions from 0-2 GeV energy range. 1,075 muons and pions were used to train the network and 1,075 $\mu/\pi$ were used as a validation set. The accuracy is how well CNN1075 is doing by epoch and was 74.5\%. The loss is gradient descent or minimization of the error of the weights and biases used in each neuron of each layer of CNN1075 and was 58\% with a trend sloping downwards on the loss curve as well as a trend sloping upward in the accuracy curve. Due to the depth of the neural network framework, it was necessary to train with a larger dataset and for more epochs, however, the downward slope of the loss curve is an indication that once trained for longer with a higher training sample, neural networks can be used for $\mu/\pi$ separation. Updates in the image making and downsampling algorithm were made to fix issues that arose in CNN1075. 
\subsection{Training CNN10000}
The hyperparameters used for CNN10000 are shown. The batch size for the training and testing as well as the test iter were chosen to encompass the whole training/testing image set when doing accuracy/loss calculations. To do this, multiplying the test iter by the test batch size give you the amount of images used when calculating accuracy/loss curves. For reference, the accuracy and loss are defined as well. 


\begin{itemize}
 \item \verb|train_batch_size: 100|
 \item \verb|test_batch_size: 100|
 \item \verb|test_iter: 100|
 \item \verb|test_interval: 100|
 \item \verb|base_lr: 0.001|
 \item \verb|lr_policy: "step"|
 \item \verb|gamma: 0.1|
 \item \verb|stepsize: 1000|
 \item \verb|display: 100|
 \item \verb|max_iter: 10000|
 \item \verb|momentum: 0.99|
 \item \verb|weight_decay: 0.0005|
 \item \verb|snapshot: 100|
\item Accuracy: How often the CNN predicts the truth over total number of images
\item Loss: Error between truth and prediction. Minimize loss by gradient descent to update weights and biases of CNN
\end{itemize}

The same architecure that was used to train CNN1075 was employed on CNN10000, Imagenet. Caffe \cite{caffe} was the software package used for both CNNs. The differences include batch size and test{\_}iter and momentum to account for the larger dataset. Both CNNs were trained on a CPU machine, Syracuse01. Further training will be done on a GPU cluster stationed at Syracuse University. Figure \ref{fig:loss_accuracy} shows the loss and accuracy of CNN10000. There is around a 10\% increase in accuracy from CNN1075 to CNN10000, 85\%, and around a 20\% decrease in loss, 36\%.      

\begin{figure}[htp!]
\centering
\includegraphics[scale=.55]{figs/acc_loss_10000_062117.png}
\caption{Accuracy vs. Loss of ImageNet 2-output $\mu/\pi$ sample consisting of ~10000 images each.} 
\label{fig:loss_accuracy}
\end{figure}

\begin{figure}[htp!]
\centering
	\begin{subfigure}[b]{.4\textwidth}
	\includegraphics[width=\textwidth,height=3in]{figs/train_confusion-9-14-16.png}
	\caption{Confusion Matrix showing Accuracy of CNN using training data}
	\label{fig:confusion}
	\end{subfigure}
	\quad
	\begin{subfigure}[b]{.4\textwidth}
	\includegraphics[width=\textwidth,height=3in]{figs/val_confusion_9-14-16.png}
	\caption{Confusion Matrix showing Accuracy of CNN using testing data}
	\label{fig:confusion_test}
	\end{subfigure}
	\quad
	\begin{subfigure}[b]{\textwidth}
	\includegraphics[width=\textwidth,height=2.5in]{figs/mitch_hw.png}
	\caption{Probability plot of muons and pions from testing set. Images surrounding histogram are a random event from lowest bin and highest bin for each particle.}
	\label{fig:prob_plot}
	\end{subfigure}
\caption{Description of confusion matrix varables: False pion rate = $false \pi/ total \pi$ True pion rate = $true \pi/total \pi$ Accuracy = $(true \pi rate + true \mu rate)/2$ Pion prediction value = $true \pi/(true \pi + false \pi)$ Muon prediction value = $true \mu/(true \mu + false \mu)$ \ref{fig:prob_plot} The probability plot includes muons and pions that are classified as primary particles.}
\label{fig:CNN_train}
\end{figure}

Figure \ref{fig:CNN_train} show a breakdown of $\mu/\pi$ separation for CNN10000. It also shows the network is not being overtrained due to the Accuracy of both the training and testing datasets being within .01\% of eachother. The CNN is doing a very good job of classifying true muons as muons, and our loss increase from CNN1075 is due to the increase in accurately classifing pions as pions. 


%\begin{figure}[htp]
%\centering
%\includegraphics[width=.9\linewidth]{Purity_numbers.png}
%\caption{ Signal and background event numbers at final selection level estimated from a BNB+Cosmic sample and Cosmic only sample normalized to $5*10^{19}$ POT. The last column gives the fraction of this signal or background type to the total 2604 selected events.} 
%\label{fig:puritytable}
%\end{figure}

\subsection{Training CNN100000}

\begin{figure}[htp]
\centering
\includegraphics[width=.8\linewidth=]{../images/confusion_allparticle_v3.pdf}
\caption{Confusion Matrix of all five particles }
\label{fig:confusion100000}
\end{figure}

\begin{figure}[htp]
\centering
\includegraphics[width=.8\linewidth=]{../images/GPU_9010split_hires_iterations_alldata_allparticle.png}
\caption{Training and testing accuracy of CNN trained on 100,000 images of \mu/\pi/p/\gamma/e with 20,000 images of each particle. Each image was a size of 576x576 and the images per particle were split 90\% use for training and 10\% used for testing the network}
\label{fig:gpuacc}
\end{figure}

\begin{figure}[htp]
\centering
\includegraphics[width=.8\linewidth=]{../images/GPU_9010split_hires_iterations_alldata_allparticle_loss.png}
\caption{Training and testing loss oc CNN trained on 100,000 images of \mu/\pi/p/\gamma/e}
\label{fig:gpuloss}
\end{figure}

\begin{figure}[htp]
\centering
\includegraphics[width=.8\linewidth=]{../csv_output/true_labels_1102.png}
\caption{t-SNE of CNN}
\label{fig:tsne}
\end{figure}

\begin{figure}[htp]
\centering
\includegraphics[width=.8\linewidth=]{../csv_output/muon_prob.png}
\caption{Muon Prob}
\label{fig:muonprob}
\end{figure}

\begin{figure}[htp]
\centering
\includegraphics[width=.8\linewidth=]{../csv_output/pi_prob.png}
\caption{Pion Prob}
\label{fig:piprob}
\end{figure}

\begin{figure}[htp]
\centering
\includegraphics[width=.8\linewidth=]{../csv_output/proton_prob.png}
\caption{Proton Prob}
\label{fig:protonprob}
\end{figure}

\begin{figure}[htp]
\centering
\includegraphics[width=.8\linewidth=]{../csv_output/eminus_prob.png}
\caption{Electron Prob}
\label{fig:eminusprob}
\end{figure}

\begin{figure}[htp]
\centering
\includegraphics[width=.8\linewidth=]{../csv_output/gamma_prob.png}
\caption{Gamma Prob}
\label{fig:gammaprob}
\end{figure}

\begin{figure}[htp]
\centering
\includegraphics[width=.8\linewidth=]{../csv_output/prob_allparticle_normalized.pdf}
\caption{Prob}
\label{fig:prob}
\end{figure}

\begin{figure}[htp]
\centering
\includegraphics[width=.8\linewidth=]{../csv_output/mu_pi_acc_tracklength_all.pdf}
\caption{mupi}
\label{fig:mu_pi}
\end{figure}

\begin{figure}[htp]
\centering
\includegraphics[width=.8\linewidth=]{../csv_output/mu_pi_acc_tracklength_75.pdf}
\caption{mupi}
\label{fig:mu_pi}
\end{figure}

\begin{figure}[htp]
\centering
\includegraphics[width=.8\linewidth=]{../csv_output/mu_pi_acc_momentum.pdf}
\caption{mupi}
\label{fig:mu_pi}
\end{figure}

\begin{figure}[htp]
\centering
\includegraphics[width=.8\linewidth=]{../csv_output/mu_p_acc_tracklength_all.pdf}
\caption{mup}
\label{fig:mu_p}
\end{figure}

\begin{figure}[htp]
\centering
\includegraphics[width=.8\linewidth=]{../csv_output/mu_p_acc_tracklength_75.pdf}
\caption{mup}
\label{fig:mu_p}
\end{figure}

\begin{figure}[htp]
\centering
\includegraphics[width=.8\linewidth=]{../csv_output/mu_p_acc_momentum.pdf}
\caption{mup}
\label{fig:mu_p}
\end{figure}

\begin{figure}[htp]
\centering
\includegraphics[width=.8\linewidth=]{../csv_output/mu_e_acc_tracklength_all.pdf}
\caption{mue}
\label{fig:mu_e}
\end{figure}

\begin{figure}[htp]
\centering
\includegraphics[width=.8\linewidth=]{../csv_output/mu_e_acc_tracklength_75.pdf}
\caption{mue}
\label{fig:mu_e}
\end{figure}

\begin{figure}[htp]
\centering
\includegraphics[width=.8\linewidth=]{../csv_output/mu_e_acc_momentum.pdf}
\caption{mue}
\label{fig:mu_e}
\end{figure}

\begin{figure}[htp]
\centering
\includegraphics[width=.8\linewidth=]{../csv_output/mu_g_acc_tracklength_all.pdf}
\caption{mug}
\label{fig:mu_g}
\end{figure}

\begin{figure}[htp]
\centering
\includegraphics[width=.8\linewidth=]{../csv_output/mu_g_acc_tracklength_75.pdf}
\caption{mug}
\label{fig:mu_g}
\end{figure}

\begin{figure}[htp]
\centering
\includegraphics[width=.8\linewidth=]{../csv_output/mu_g_acc_momentum.pdf}
\caption{mug}
\label{fig:mu_g}
\end{figure}


  \chapter{Using Convolutional Neural Networks for $\nu_{\mu}$ CC event classification}\label{ch:cnn_results}
\section{Classification using CNN10000}

%-----------------------------Commenting out selection I original work------------------------------------------------------------------
\begin{comment}
\subsection{Classification of MC data using Selection I Original CC-Inclusive Filter}\label{sel1orig}

\begin{figure}[htp!]
\centering
\includegraphics[width=.9\textwidth]{figs/sel1_cuts.png}
\caption{Snapshot of passing rates of Selection I from CC-Inclusive Filter} 
\label{fig:cuttable}
\end{figure}

\begin{figure}[htp!]
\centering
	\begin{subfigure}[b]{.45\textwidth}
	\includegraphics[width=3in,height=3in]{figs/confusion_0621_wrongnorm.png}
	\caption{Confusion Matrix showing Accuracy of CNN using data with wrong normilazion}
	\label{fig:confusion_wrongnorm}
	\end{subfigure}
	\quad
	\begin{subfigure}[b]{.45\textwidth}
	\includegraphics[width=3in,height=3in]{figs/prob_0706_wrongnorm_sel1.png}
	\caption{Probability plot showing $\mu/\pi$ separation of CNN using wrong normalization}
	\label{fig:prob_wrongnorm}
	\end{subfigure}
	\quad
	\begin{subfigure}[b]{.45\textwidth}
	\includegraphics[width=3in,height=3in]{figs/confusion_rightnorm_0621.png}
	\caption{Confusion Matrix showing Accuracy of CNN using data with correct normilazion}
	\label{fig:confusion_rightnorm}
	\end{subfigure}
	\quad
	\begin{subfigure}[b]{.45\textwidth}
	\includegraphics[width=3in,height=3in]{figs/prob_0706_rightnorm_sel1.png}
	\caption{Probability plot showing $\mu/\pi$ separation of CNN using correct normalization}
	\label{fig:prob_rightnorm}
	\end{subfigure}
	\quad
\caption{Results of CNN10000 classification of track candidate images output from cc-inclusive filter.}
\label{fig:CNN_ccnc}
\end{figure}

\begin{figure}[htp!]
\centering
	\begin{subfigure}[b]{.45\textwidth}
	\includegraphics[width=3in,height=3in]{figs/sel1_trackrange_wrongnorm_acc70_0706.png}
	\caption{Track range distribution of events from Selection I Original passing CNN with 70\% accuracy using image data with wrong normilazion}
	\label{fig:track_wrongnorm}
	\end{subfigure}
	\quad
	\begin{subfigure}[b]{.45\textwidth}
	\includegraphics[width=3in,height=3in]{figs/sel1_trackrange_rightnorm_acc70_0706.png}
	\caption{Track range distribution of events from Selection I Original passing CNN with 70\% accuracy using image data with correct normilazion}
	\label{fig:track_rightnorm}
	\end{subfigure}
	\quad
	\begin{subfigure}[b]{.45\textwidth}
	\includegraphics[width=3in,height=3in]{figs/sel1_parP_wrongnorm_acc70_0706.png}
	\caption{Momentum distribution of events from Selection I Original passing CNN with 70\% accuracy using image data with wrong normilazion}
	\label{fig:momentum_wrongnorm}
	\end{subfigure}
	\quad
	\begin{subfigure}[b]{.45\textwidth}
	\includegraphics[width=3in,height=3in]{figs/sel1_parP_rightnorm_acc70_0706.png}
	\caption{Momentum distribution of events from Selection I Original passing CNN with 70\% accuracy using image data with correct normilazion}
	\label{fig:momentum_rightnorm}
	\end{subfigure}
	\quad
\caption{CNN10000 distributions of track candidate images output from Selection I Original cc-inclusive filter with different image data normalizations}
\label{fig:CNN_dist}
\end{figure}

The next step that was taken was to use CNN10000 to classify track candidate images that were identified by the Selection I original cc-inclusive filter described in \cite{cc-inclusive}. Passing rates for each cut in cc-inclusive filter are show in figure \ref{fig:cuttable}. For the incorrect image making normalization dataset, out of 188,880 events, 7438 passed the cut right before 75 cm track length cut which is 3.9\% of total data. Discrepancies in passing rates are due to grid submission issues, however, this dataset is used to check if changes in image making normalization affects $\mu/\pi$ separation probability due to CNN10000 being trained with incorrectly image making normalized data. For the second dataset with correct image making normalization, out of 188,880 events, 9552 events passed the cut right before the 75 cm track length cut which is 5.1\% passing rate and is comparable to figure \ref{fig:cuttable}. Intime cosmics were also run over for efficiency and purity calculations. Out of 14395 in time cosmic events, 175 passed the cut right before the 75 cm track length cut which is a passing rate of 1.2\% compared to 1.3\% shown in table \ref{table:mc} in section \ref{section:eventselection} . 

Figures \ref{fig:confusion_wrongnorm}, \ref{fig:prob_wrongnorm}, \ref{fig:confusion_rightnorm} and \ref{fig:prob_rightnorm} show the accuracy and $\mu/\pi$ separation of both the correct and incorrect normalized images. The confusion matrices are only composed of $\mu/\pi$ data. Other particles passed the cc-inclusive filter before the 75 cm track length cut and were all mis-id'ed as muons. Since CNN10000 has not seen any particles other than muons and pions, it makes sense that those get mis-id'ed. Figures \ref{fig:prob_wrongnorm} and \ref{fig:prob_rightnorm} don't have $\mu/\pi$separation comparable to \ref{fig:prob_plot}, but \ref{fig:prob_wrongnorm} does skew to higher probabilities compared to \ref{fig:prob_rightnorm}. This is to be expected and further work on quantifying the performance of CNN10000 should use the incorrect image making normalization. It is also expected that the separation isn't as defined as the testing dataset for CNN10000. CNN10000 was trained and tested using single particle muons and pions and the track candidate dataset come from BNB+Cosmic events, not to mentions all track candidates have passed the cc-inclusive filter that tags "muon-like" tracks therefore the pions in this sample look much closer in muon topology than the network has seen. Also, these images were made from wire and time ticks associated to hits from the track candidate that passed the cc-inclusive filter. This is different from the training images where a bounding box was drawn over the total $\mu$ or $\pi$ interaction. Spurious energy deposition from a $\pi-Ar$ interaction is most likely not included in the BNB+Cosmic images due to the tracking algorithm. To remedy this, the CNN needs to see more "muon-like" pions and muons and pions from a neutrino interaction passing the cc-inclusive filter as well as a larger particle variety including protons, photons and electrons. Although $\mu/\pi$ separation is lacking, CNN10000 does an excellent job of classifying muons and using higher CNN probability can increase purity. Figures \ref{fig:track_wrongnorm}, \ref{fig:track_rightnorm}, \ref{fig:momentum_wrongnorm} and \ref{fig:momentum_rightnorm} show the track and momentum distributions for these two datasets. In both sets you have an increase in data in the bin below 75 cm and at bins below 0.5 GeV. These distributions were made with events classified with 70\% probability of being a muon regardless of true particle type. 
\end{comment}
%-----------------------------Commenting out selection I original work------------------------------------------------------------------

\subsection{Classification of MC data using Selection I CC-Inclusive Filter}

After training CNN10000, it was then used to classify track candidate images that were identified by the Selection I cc-inclusive filter right before the 75 cm track length cut described in chapter \ref{ch:meas}. Passing rates for each cut in this filter are shown in table \ref{table:mc}. Out of 188,880 events, 19,112 passed the cut right before the 75 cm track length cut which is a 10.1\% passing rate and comparable to the 10\% passing rate shown in table \ref{table:mc}. Intime cosmics were also run over, out of 14,606 in time cosmics events, 302 passed the cut right before the 75 cm track length cut which is a 2.1\% passing rate comparable to the 2.7\% passing rate in the cc-inclusive tech-note. Figures \ref{fig:confusion_sel1mod} and \ref{fig:prob_sel1mod} show the accuracy and $\mu/\pi$ separation. Both plots are only composed of muons and pions due to the focus on $\mu/\pi$ separation and the fact that CNN10000 was only trained on muons and pions, however, for reference, all other particles that did pass Selection I were mis-id'ed as muons. Muons are being identified at a very high rate, while pions are all being mis-id'ed as muons. This is due in part because the pion track candidate that does pass the cc-inclusive filter right before the 75 cm track length cut has already been identified as a muon candidate, hence, at a higher muon probability. Another reason for the pion mis'id can be attributed to the training/classifying dataset difference. For training, the pion images include the whole pion interaction in argon, including any decays or nucleon scattering. The image created from a BNB+Cosmic event used for classification only includes the track candidate that passed the cc-inclusive filter right before the 75 cm track length cut.
Figure \ref{fig:sel1mod_track} shows the track range distributions of all events from Selection I being classified by the CNN as a muon with a probability of 70\% regardless of true particle type. We get entries for the CNN curve in the lowest bin and none for the 75 cm curve. To see how many true CC events were identified by CNN10000 breaking down figure \ref{fig:sel1mod_track} by event type was necessary. Figures \ref{fig:sel1mod_stackedcnn} and \ref{fig:sel1mod_stackedoriginal} show track range distributions separated by signal and various backgrounds. Particle type was not taken into consideration in these plots so true CC event images can be any track candidate particle passing Selection I cut right before track length cut including pions and protons. 

To gain an even deeper understanding on how CNN10000 is performing, plotting these distributions with only muons and pions was done due to the fact that CNN10000 was trained with only those particles for $\mu/\pi$ separation. Figures \ref{fig:sel1mod_mupi_70stackedcnn}-\ref{fig:sel1mod_mupi_90stackedcnn} show the stacked histograms of signal and background of the track range distributions with varying CNN probabilities starting from 70\% and ending at 90\% probability. With higher probabilities we get a purer sample in the lower bin but we end up losing events as well. Momentum distributions for all signal/background events are shown in figure \ref{fig:sel1mod_parP}. At CNN10000 at 70\% we introduce more NC background, however, we also get more CC events passing as well.   

\begin{figure}[htp!]
\centering
	\begin{subfigure}[b]{.45\textwidth}
	\includegraphics[width=3in,height=3in]{figs/sel1mod_confusion_wrongnorm.png}
	\caption{Confusion Matrix for CNN10000 classified events from Selection I}
	\label{fig:confusion_sel1mod}
	\end{subfigure}
	\quad
	\begin{subfigure}[b]{.45\textwidth}
	\includegraphics[width=3in,height=3in]{figs/probplot_wrongnorm_selImod.png}
	\caption{Probability plot for CNN10000 classified events from Selection I}  
	\label{fig:prob_sel1mod}
	\end{subfigure}
	\quad
\caption{Confusion matrix and probability plot of events passing Selection I cc-inclusive cuts right before 75cm track length cut}
\label{probplots}
\end{figure}

\begin{figure}[htp!]
\centering
	\begin{subfigure}[b]{.9\textwidth}
	\centering
	\includegraphics[width=4in,height=2.5in]{figs/sel1mod_trackrange_wrongnorm_acc70_0706.png}
	\caption{Track range distribution of events from Selection I passing CNN with 70\% accuracy}
	\label{fig:sel1mod_track}
	\end{subfigure}
	\quad
	\begin{subfigure}[b]{.45\textwidth}
	\includegraphics[width=\textwidth, height=2in]{figs/sel1mod_cnn_stackedevent_0707.png}
	\caption{Stacked signal and background track range distributions from Selection I passing CNN with 70\% accuracy}
	\label{fig:sel1mod_stackedcnn}
	\end{subfigure}
	\quad
	\begin{subfigure}[b]{.45\textwidth}
	\includegraphics[width=\textwidth, height=2in]{figs/sel1mod_original_stackedevents_0707.png}
	\caption{Stacked signal and background track range distributions from Selection I passing 75 cm track length cut}
	\label{fig:sel1mod_stackedoriginal}
	\end{subfigure}
	\quad
	\begin{subfigure}[b]{.45\textwidth}
	\includegraphics[width=\textwidth, height=2in]{figs/sel1mod_cnn_trackrange_mupi_acc70_0707.png}
	\caption{Stacked signal muons and background muons/pions of track range distributions from Selection I passing CNN with 70\% accuracy}
	\label{fig:sel1mod_mupi_70stackedcnn}
	\end{subfigure}
	\quad
	\begin{subfigure}[b]{.45\textwidth}
	\includegraphics[width=\textwidth, height=2in]{figs/sel1mod_original_trackrange_mupi_acc70_0707.png}
	\caption{Stacked signal muons and background muons/pions of track range distributions from Selection I passing 75 cm track length cut}
	\label{fig:sel1mod_mupi_70stackedoriginal}
	\end{subfigure}
	\quad
\caption{CNN10000 distributions of track candidate images output from Selection I cc-inclusive filter}
\label{fig:sel1mod_CNN_dist}
\end{figure}



\begin{figure}[htp!]
\centering
	\begin{subfigure}[b]{.45\textwidth}
	\includegraphics[width=\textwidth, height=2.5in]{figs/sel1mod_cnn_trackrange_acc75_0707.png}
	\caption{Stacked signal muons and background muons/pions of track range distributions from Selection I passing CNN with 75\% accuracy}
	\label{fig:sel1mod_mupi_75stackedcnn}
	\end{subfigure}
	\quad
	\begin{subfigure}[b]{.45\textwidth}
	\includegraphics[width=\textwidth, height=2.5in]{figs/sel1mod_cnn_trackrange_acc80_0707.png}
	\caption{Stacked signal muons and background muons/pions of track range distributions from Selection I passing CNN with 80\% accuracy}
	\label{fig:sel1mod_mupi_80stackedcnn}
	\end{subfigure}
	\quad
	\begin{subfigure}[b]{.45\textwidth}
	\includegraphics[width=\textwidth, height=2.5in]{figs/sel1mod_cnn_trackrange_acc85_0707.png}
	\caption{Stacked signal muons and background muons/pions of track range distributions from Selection I passing CNN with 85\% accuracy}
	\label{fig:sel1mod_mupi_85stackedcnn}
	\end{subfigure}
	\quad
	\begin{subfigure}[b]{.45\textwidth}
	\includegraphics[width=\textwidth, height=2.5in]{figs/sel1mod_cnn_trackrange_acc90_0707.png}
	\caption{Stacked signal muons and background muons/pions of track range distributions from Selection I passing CNN with 90\% accuracy}
	\label{fig:sel1mod_mupi_90stackedcnn}
	\end{subfigure}
	\quad
\caption{CNN10000 stacked signal/background track range distributions of track candidate images output from Selection I cc-inclusive filter}
\label{fig:sel1modCNNdistacc}
\end{figure}


\begin{figure}[htp!]
\centering
	\begin{subfigure}[t]{.9\textwidth}
	\centering
	\includegraphics[width=\textwidth,height=3.5in]{figs/sel1mod_parP_wrongnorm_acc70_0706.png}
	\caption{Momentum distribution of events from Selection I passing CNN with 70\% accuracy}
	\label{fig:sel1mod_momentum}
	\end{subfigure}
	\quad
	\begin{subfigure}[t]{.45\textwidth}
	\includegraphics[width=\textwidth,height=2.5in]{figs/sel1mod_cnn_parP_stackedevents_0707.png}
	\caption{Stacked signal and background momentum distributions from Selection I passing CNN with 70\% accuracy}
	\label{fig:sel1mod_momentum_stackedcnn}
	\end{subfigure}
	\quad
	\begin{subfigure}[t]{.45\textwidth}
	\includegraphics[width=\textwidth,height=2.5in]{figs/sel1mod_original_parP_stackedevents_0707.png}
	\caption{Stacked signal and background momentum distributions from Selection I passing 75 cm track length cut}
	\label{fig:sel1mod_momentum_stackedoriginal}
	\end{subfigure}
	\quad
\caption{CNN10000 momentum distributions of track candidate images output from Selection I cc-inclusive filter}
\label{fig:sel1mod_parP}
\end{figure}

Another check was to see if any true CC pions were passing through the cut right before the 75 cm track length cut. Figure \ref{fig:mupi} shows the comparison of the stacked track range distribution with only true CC muon signal versus the stacked distribution with true CC muons and pions signal. As you can see, we gain more events when plotting CC events with a particle type of either muons or pions due to the CNN classifying all pions in this dataset as muons. This is an interesting scenario and a sample of topologies of these images are represented in figure \ref{fig:evd}, at least 3 tracks are coming out of the vertex for these types of events. With the 75 cm track length cut, the selection is cutting event topologies like this where the pion is the tagged track candidate. Figure \ref{fig:longer_muon_badreco} has a defined longer muon track, but because of dead wires through the track, the reconstructed range is 1. less than 75 cm and 2. shorter than the reconstructed pion whose length is also less than 75 cm. This is a very interesting event, but because of issues with the tracking algorithm, the 75 cm cut would get rid of this event. The CNN was able to recover this event only because it has classified all pions as muons. Figure \ref{fig:longer_pion} shows the second case to think about, the pion, while still less than 75 cm has a reconstructed track length longer than the muon. Again, the CNN recovered this event due to pions being classified as muons. Lastly, figure \ref{fig:verylongpion} shows a pion with a reconstructed track length greater than 75 cm and the muon. These three cases show that a broader question must be asked when training the network other than is it a muon or pion. There are different routes to recover interesting events like these. One route is to ask the network ``Is it a CC event or is it an NC event?'' and obtain an image dataset consisting of whole CC/NC events that will train the network to answer this question. The other route is to ask the network ``Is this a $\mu/\pi/p/$ from a CC event or NC event and obtain an image dataset consisting of primary particles from a CC/NC event. Both these paths will be explored in future work.     

\begin{figure}[htp!]
\centering
	\begin{subfigure}[b]{.45\textwidth}
	\includegraphics[width=\textwidth,height=2.5in]{figs/sel1mod_cnn_trackrange_mupi_acc70_0707.png}
	\caption{Stacked signal $\mu$/background $\mu$ and $\pi$ track range distribution of CNN @ 70\%}
	\end{subfigure}
	\quad
	\begin{subfigure}[b]{.45\textwidth}
	\includegraphics[width=\textwidth,height=2.5in]{figs/sel1mod_mupi_trackrange_acc70.png}
	\caption{Stacked signal $\mu \& \pi$/background $\mu \& \pi$ track range distribution of CNN @ 70\%}
	\label{fig:mupib}
	\end{subfigure}
	\quad
\caption{Track distribution comparisons of true CC muons plotted vs true CC muons and pions plotted}
\label{fig:mupi}
\end{figure}


\begin{figure}[htp!]
\centering
	\begin{subfigure}[b]{.3\textwidth}
	\includegraphics[width=\textwidth,height=2.5in]{figs/interesting_event.png}
	\caption{Pion reconstructed track range is less than 75 cm and longer than muon track due to dead wires}
	\label{fig:longer_muon_badreco}
	\end{subfigure}
	\quad
	\begin{subfigure}[b]{.3\textwidth}
	\includegraphics[width=\textwidth,height=2.5in]{figs/event2.png}
	\caption{Pion reconstructed track range is less than 75 cm and larger than muon reconstructed track}
	\label{fig:longer_pion}
	\end{subfigure}
	\quad
	\begin{subfigure}[b]{.3\textwidth}
	\includegraphics[width=\textwidth,height=2.5in]{figs/mupievent.png}
	\caption{Pion reconstructed track range is greater than 75 cm and larger than muon reconstructed track}
	\label{fig:verylongpion}
	\end{subfigure}
	\quad
\caption{Images of true CC events where the pion was the tagged track candidate}
\label{fig:evd}
\end{figure}

\begin{table}[htp!]
\centering
\resizebox{\textwidth}{!}{ \begin{tabular}{c||c| c c| c| c} % centered columns (4 columns)
\hline %inserts double horizontal lines
\toprule 
      && \multicolumn{2}{c}{BNB + Cosmics} & \multicolumn{1}{c}{Cosmic Only} & \multicolumn{1}{c}{Signal:}\\

      && Selection & MC Truth & &Cosmic Only \\

\midrule
75 cm Cut passing rates	& Generated Events  	       & 191362             & 45723               & 4804                & 1:22 \\ % inserting body of the table
                        & Track Containment 	       & 19391 (48\%/10\%)  & 11693 (45\%/26\%)   & 129 (38\%/2.7\%)    & 1:2.3  \\
\rowcolor{LightCyan}    & track $\geq$ 75 cm 	       & 6920 (36\%/3.6\%)  & 5780 (49\%/13\%)    & 17 (13\%/0.4\%)     & 1:0.6  \\
\hline\hline
CNN passing rates       & Generated Events  	       & 188880             & 44689               & 14606               & 1:21 \\ % inserting body of the table
		        & Track Containment 	       & 19112 ( /10\%)     & 11554 ( /26\%)      & 302 ( /2.1\%)       & 1:1.73  \\
\rowcolor{LightCyan}    & CNN cut @ 70\% Probability   & 16502 (86\%/8.7\%) & 10605 (92\%/23\%)   & 205 (68\%/14\%)     & 1:1.28  \\
\rowcolor{LightCyan}    & CNN cut @ 83\% Probability   & 7511 (46\%/4.0\%)  & 6142  (58\%/14\%)   & 32 (16\%/0.2\%)     & 1:0.4  \\
\bottomrule
\hline %inserts single line
\end{tabular}}
\caption{Comparing passing rates of CNN at different probabilities versus 75 cm track length cut: Numbers are absolute event counts and Cosmic background is not scaled appropriately. The BNB+Cosmic sample contains all events. The numbers in brackets give the passing rate wrt the step before (first percentage) and wrt the generated events (second percentage). In the BNB+Cosmic MC Truth column shows how many true $\nu_{\mu}$ CC-inclusive events (in FV) are left in the sample. This number includes possible mis-identifications where a cosmic track is picked by the selection instead of the neutrino interaction in the same event.The CNN MC True generated events were scaled wrt the MC True generated events for the 75 cm cut passing rates due to only running over 188,880 generated events versus the 191362 generated events. The last column Signal:Cosmic only gives an estimate of the $\nu_{\mu}$ CC events wrt the cosmic only background at each step. For this number, the cosmic background has been scaled as described in \cite{cc-inclusive}. Note that these numbers are not a purity, since other backgrounds can’t be determined at this step.} 
% title of Table
\label{table:passingrates} % is used to refer this table in the text
\end{table}


\begin{table}[htp!]
\centering
\resizebox{\textwidth}{!}{ \begin{tabular}{c c c|a} % centered columns (4 columns)
%\hline %inserts double horizontal lines
      &&\#Events(Fraction)  & \#Events(Fraction) \\
      &&passing Sel I & passing CNN @ 83\% Probability\\


Signal	       & $\nu_{\mu}$ CC events with true vertex in FV      & 1168(53.8\%)       & 6142(61\%)\\ % inserting body of the table
\hline\hline %inserts double horizontal lines
Backgrounds    & Cosmics Only Events                               & 725(33.4\%)       & 2582(26\%)\\ % inserting body of the table
               & Cosmics in BNB Events                             & 144(6.6\%)       & 492(4.9\%)\\ % inserting body of the table
               & NC Events                                         & 75(3.5\%)        & 778(7.7\%)\\ % inserting body of the table
               & $\nu_e$ and $\bar{\nu}_e$ Events                  & 4(0.2\%)        & 32(0.3\%)\\ % inserting body of the table
               & $\bar{\nu}_{\mu}$  Events                         & 40(1.8\%)         & 67(0.7\%)\\ % inserting body of the table
\end{tabular}}
\caption{Signal and background event numbers of Selection I and Selection I with CNN cut estimated from a BNB+Cosmic sample and Cosmic only sample normalized to $5*10^{19}$ PoT. The last column gives the fraction of this signal or background type to the total selected events per CNN probability.} 
\label{table:purity} % is used to refer this table in the text
\end{table}

Table \ref{table:passingrates} shows the passing rates for the 75 cm track length cut and the CNN cut at 70\% and 83\%. The passing rates at the track containment level for the 75 cm track length cut compared to the CNN are comparable with only a 0.6\% difference in the in time cosmic bin which may be due in part to the larger in time cosmic statistics used for the CNN dataset. These passing rates need to be comparable to then be able to compare the passing rates after the CNN cut to the 75 cm cut. Again, the same BNB+Cosmic sample was used for both Selection I with 75 cm cut and Selection I with CNN cut. As it stands, a CNN cut at 83\% probability has a MC true CC event passing rate of 14\% compared to the 13\% passing rate of the 75 cm track length cut. The Signal:Cosmic Only background is also reduced from 1:0.6 to 1:0.4 The total passing rate is also higher than the 75 cm cut, 3.6\% vs 4.0\%. Table \ref{table:purity} shows the breakdown of signal and backgrounds for the CNN at the different probabilities. We have a 61\% signal passing rate with the CNN cut @ 83\% versus the 53.8\% signal passing rate of the 75 cm cut. 

Based on these numbers, the following performance values of the selection with 75 cm cut versus selection with CNN @ 83\% probability cut were calculated:
\begin{itemize}
\item Efficiency: Number of selected true $\nu_{\mu}$ CC events divided by the number of expected true $\nu_{\mu}$ CC events with interaction in the FV.
\begin{itemize}
\item Selection I: 12.3\% 
\item Selection I with CNN10000 cut @ 83\% probability: 14\% 
\end{itemize}
\item Purity: Number of selected true $\nu_{\mu}$ CC events divided by sum of itself and the number of all backgrounds.
\begin{itemize}
\item Selection I: 53.8\% 
\item Selection I with CNN10000 cut @ 83\% probability: 61\% 
\end{itemize}
\end{itemize}

Lastly, figure \ref{fig:sel1mod_cnnperformance} shows a more representative performance of the CNN. Due to the fact that the CNN was trained on muons and pions, showing the performance of CC muon events versus NC pion events with respect to CNN probability gives a better picture of how the network is performing. Figure \ref{fig:sel1mod_cnnperformance} shows that at 83\% we are below the 75 cm cut NC pion threshold and still above the CC muon threshold. Using 83\% probability not only reduced the NC pion background, it also dramatically reduced the intime cosmics and cosmics in the BNB. Figure \ref{fig:sel1mod_83} shows the track range of signal muons and pions compared to background muons and pions from cosmic rays or NC interactions. Comparing figure \ref{fig:sel1mod_83} to \ref{fig:mupib} you can see the reduction in both the cosmic and NC backgrounds.  

\begin{figure}[htp!]
\centering
\includegraphics[width=3in,height=2.5in]{figs/cnn_performance.png}
\caption{CNN performance of classified muons and pions compared to the already implemented 75 cm track length cut}
\label{fig:sel1mod_cnnperformance}
\end{figure}

\begin{figure}[htp!]
\centering
\includegraphics[width=3in,height=2.5in]{figs/sel1mod_trackrange_acc83.png}
	\caption{Stacked signal $\mu \& \pi$/background $\mu \& \pi$ track range distribution of CNN @ 83\%}
\label{fig:sel1mod_83}
\end{figure}
\subsection{Conclusions of CNN10000 classification of MC data}

It was shown that even though CNN10000 was trained with single particle generated muons and pions, it performs fairly well at classifying track candidate images from BNB+Cosmic events. Events have been regained below the 75 cm track length cut and the momentum and track range distributions have similar shapes to the distributions of Selection I. Efficiencies and purities were calculated for Selection I events before 75 cm track length cut  with the CNN at 83\% probability and are 14\% and 62\% respectively. Although the CNN doesn't have separation between muons and pions and although all particles passing CNN are classified as muon, increasing CNN probability allows us to increase the purity as well as maintain an efficiency comparable to the 75 cm track length cut all while recovering events below that 75 cm cut. Out of the 6142 events that passed the CNN @ 83\% 1470 events were below the 75 cm cut, a recovery of 3.3\% of data excluded by the 75 cm cut with a purity of 15\% which is better than Selection I. Although these numbers are low, it is an improvement from the selecion I in both total efficiency and purity and an increase in phase space by recovering these events. 

\section{Classification using CNN100000}
For classification of BNB+Cosmics and data using CNN100000, images were made from track candidates that passed the Selection I filter, however, unlike for classifying BNB+Cosmics using CNN10000, the classification of CNN100000 went further up Selection I's cut chain. For CNN100000, steps 5 through 8 seen in section \ref{section:eventselection} were removed. The image making algorithm would then create multiple images per event of pixels corresponding to each track associated to the flattest vertex candidate in the fiducial volume. One of the findings of CNN10000 was the possibility of recovering interesting events in which a pion from a cc-inclusive event is tagged as the track candidate of interest. This was the reason for trying to expand on what a convolutional neural network could accomplish. By allowing the CNN to particle ID all track associated with the vertex candidate, we allow the selection to contain the interesting events that were cut out in Selection I due to the cc pion track being chosen as the track candidate. Figure \ref{fig:cnn100000_image} shows the image making algorithm for BNB+Cosmic images. The classification algorithm would then particle ID each image in an event. If at least one of the images is identified as a muon by the CNN, the event is then classified as a $\nu_{\mu}$ event. The image with the highest muon probability is then chosen to be the track candidate and used for kinematic distribution purposes. The CNN's selection and efficiency can be tuned by increasing the muon probability of the muon track candidate image. The results of using CNN100000 to classify BNB+Cosmics will be discussed in the next sections. 

\begin{figure}[htp!]
\centering
\includegraphics[width=\textwidth]{figs/cnn100000_image.png}
\caption{Image making steps used for classifying BNB+Cosmic events using CNN100000} 
\label{fig:cnn100000_image}
\end{figure}

\subsection{Classification of MC data using Selection I CC-Inclusive Filter}
After classifying all BNB+Cosmic and in time cosmic events, an efficiency vs purity curve was created for various muon probabilities to choose a probability that would increase both efficiency and purity of Selection I. This is shown in figure \ref{fig:roc}. Selection I and Selection II are also shown on this curve. At 85\% probability, both the efficiency and purity is better than both Selection I and Selection II therefore is the chosen muon probability. Although the efficiency and purity of CNN100000 have a vast increase from Selection I, making sure the truth kinematic distributions between the two selections is an important thing to check to make sure one selection isn't focusing on a different phase space than the other. Also applying CNN100000 to data to see if it responds similarly is important. 

\begin{figure}[htp!]
\centering
\includegraphics[width=.5\textwidth]{figs/roc_cnn_selI&II.png}
\caption{Efficiency vs Purity curve for various CNN100000 muon probabilities. At 85\% muon probability, the efficiency is 30\% and the purity is 70\%} 
\label{fig:roc}
\end{figure}

Figure \ref{fig:truthkinematics} are the true kinematic distributions for the true cc-inclusive events that passed the CNN100000 (blue) at 85\% muon probability as well as the cc-inclusive events that passed the Selection I filter (red).  

\begin{figure}[htp!]
\centering
	\begin{subfigure}[b]{.475\textwidth}
	\centering
		\includegraphics[width=\textwidth]{../bnbcosmic_output/cnn_85_trackrangedist.png}
		\caption{Track range distribution for events passing CNN10000 $\geq 85\%$ and the Selection I filter.} 
		\label{fig:cnn85trackrange}
	\end{subfigure}
	\quad
	\begin{subfigure}[b]{.475\textwidth}
	\centering
		\includegraphics[width=\textwidth]{../bnbcosmic_output/cnn_85_truecosthetadist.png}
		\caption{$Cos(\theta)$ distribution for events passing CNN100000 $\geq 85\%$ and the Selection I filter.} 
		\label{fig:cnn85costheta}
	\end{subfigure}
	\quad
	\begin{subfigure}[b]{.475\textwidth}
	\centering
		\includegraphics[width=\textwidth]{../bnbcosmic_output/cnn_85_truephidist.png}
		\caption{$\phi$ distribution for events passing CNN100000 $\geq 85\%$ and the Selection I filter.} 
		\label{fig:cnn85phi}
	\end{subfigure}
	\quad
	\begin{subfigure}[b]{.475\textwidth}
	\centering
		\includegraphics[width=\textwidth]{../bnbcosmic_output/cnn_85_momentumdist.png}
		\caption{Momentum distribution for events passing CNN100000 $\geq 85 \%$ and the Selection I filter.} 
		\label{fig:cnn85momentum}
	\end{subfigure}
\caption{Truth kinematic distributions of events passing CNN100000 and Selection I. The red corresponds to the Selection I passing events and blue to the CNN100000 passing events.}
\label{fig:truthkinematics}
\end{figure}

The shapes of the true kinematic distributions are mostly comparable for CNN100000 and Selection I, however the CNN100000 curve has more events passing at muon probability 85\% compared to the Selection I filter. This is due to the removal of the containment cut. You can also see entries for cc-inclusive events at the lowest track range bin for CNN100000 that isn't there for the Selection I filter. Although the muon probability is high, we are still able to recover events with low track candidate track range. Another thing to note in the track range plot is that the peak for CNN100000 is shifted to a higher track range compared to Selection I. This peak shift can also be seen in the momentum distribution plot. Again, a hypothesis for this shift is due to the lack of containment for the tracks, this has been explored and will be discussed. 

\begin{figure}[htp!]
\centering
	\begin{subfigure}[b]{.475\textwidth}
	\centering
		\includegraphics[width=\textwidth]{../bnbcosmic_output/cc_trackrangedist_stacked.png}
		\caption{Track range distribution for events passing Selection I filter.} 
		\label{fig:cctrackstacked}
	\end{subfigure}
	\quad
	\begin{subfigure}[b]{.475\textwidth}
	\centering
		\includegraphics[width=\textwidth]{../bnbcosmic_output/cnn_85_trackrangedist_stacked.png}
		\caption{Track range distribution for events passing CNN100000 $\geq 85\%$.} 
		\label{fig:cnn85trackstacked}
	\end{subfigure}
\caption{Truth stacked event type track range distribution of events passing Selection I (left) and CNN100000 (right). Different event types are CCQE (green), CCRES (blue), CCDIS (yellow), CCCOH (red).}
\label{fig:trackstacked}
\end{figure}

\begin{figure}[htp!]
\centering
	\begin{subfigure}[b]{.475\textwidth}
	\centering
		\includegraphics[width=\textwidth]{../bnbcosmic_output/cc_momentumdist_stacked.png}
		\caption{Momentum distribution for events passing Selection I.} 
		\label{fig:ccmomentumstacked}
	\end{subfigure}
	\quad
	\begin{subfigure}[b]{.475\textwidth}
	\centering
		\includegraphics[width=\textwidth]{../bnbcosmic_output/cnn_85_momentumdist_stacked.png}
		\caption{Momentum distribution for events passing CNN100000 $\geq 85\%$.} 
		\label{fig:cnn85momentumstacked}
	\end{subfigure}
\caption{Truth stacked event type momentum distribution of events passing Selection I (left) and CNN100000 (right). Different event types are CCQE (green), CCRES (blue), CCDIS (yellow), CCCOH (red).}
\label{fig:momentumstacked}
\end{figure}

\begin{figure}[htp!]
\centering
	\begin{subfigure}[b]{.475\textwidth}
	\centering
		\includegraphics[width=\textwidth]{../bnbcosmic_output/cc_costhetadist_stacked.png}
		\caption{$Cos(\theta)$ distribution for events passing Selection I filter.} 
		\label{fig:cccosthetastacked}
	\end{subfigure}
	\quad
	\begin{subfigure}[b]{.475\textwidth}
	\centering
		\includegraphics[width=\textwidth]{../bnbcosmic_output/cnn_85_costhetadist_stacked.png}
		\caption{$Cos(\theta)$ distribution for events passing CNN100000 $\geq 85\%$.} 
		\label{fig:cnn85costhetastacked}
	\end{subfigure}
\caption{Truth stacked event type $cos(\theta)$ distribution of events passing Selection I (left) and CNN100000 (right). Different event types are CCQE (green), CCRES (blue), CCDIS (yellow), CCCOH (red).}
\label{fig:costhetastacked}
\end{figure}

\begin{figure}[htp!]
\centering
	\begin{subfigure}[b]{.475\textwidth}
	\centering
		\includegraphics[width=\textwidth]{../bnbcosmic_output/cc_phidist_stacked.png}
		\caption{$\phi$ distribution for events passing Selection I.} 
		\label{fig:ccphistacked}
	\end{subfigure}
	\quad
	\begin{subfigure}[b]{.475\textwidth}
	\centering
		\includegraphics[width=\textwidth]{../bnbcosmic_output/cnn_85_phidist_stacked.png}
		\caption{$\phi$ distribution for events passing CNN100000 $\geq 85\%$.} 
		\label{fig:cnn85phistacked}
	\end{subfigure}
\caption{Truth stacked event type phi distribution of events passing Selection I (left) and CNN100000 (right). Different event types are CCQE (green), CCRES (blue), CCDIS (yellow), CCCOH (red).}
\label{fig:phistacked}
\end{figure}
Figures \ref{fig:trackstacked} through \ref{fig:phistacked} compare the stacked event type distributions between the Selection I filter and the CNN100000. The percentage of CCQE events that passed Selection I is 51\% compared to 62\% for CNN100000. For CCRES, Selection I had a passing rate of 37\% compared to 29\% for CNN100000. The CCDIS passing rate was 11\% and 9\% for Selection I and CNN100000 respectively. Lastly, the CCCOH rate was 1\% and 0.9\% for Selection I and CNN100000 respectively. A larger percentage of CCQE events are passing the CNN100000 compared to Selection I, an increase of 9\%. CCQE is the dominant interaction for $\nu_{\mu}$ events with neutrino energy $<1$ GeV, so being able to recover events below Selection I's 75 cm track range cut may be the reason for the increase in CCQE events. 
  
Figure \ref{fig:vertex} shows the vertex positions for the true cc-inclusive events passing the CNN100000 (blue) and the Selection I (red) filter. Again, the shape distributions are comparable among the two selections other than a higher passing rate due to the lack of track containment for CNN100000.  
\begin{figure}[htp!]
\centering
	\begin{subfigure}[b]{.55\textwidth}
	\centering
		\includegraphics[width=\textwidth]{../bnbcosmic_output/cnn_85_trackstartXdist.png}
		\caption{X Vertex Position} 
		\label{fig:cnn85vertexX}
	\end{subfigure}
	\quad
	\begin{subfigure}[b]{.55\textwidth}
	\centering
		\includegraphics[width=\textwidth]{../bnbcosmic_output/cnn_85_trackstartYdist.png}
		\caption{Y Vertex Position} 
		\label{fig:cnn85vertexY}
	\end{subfigure}
	\quad
	\begin{subfigure}[b]{.55\textwidth}
	\centering
		\includegraphics[width=\textwidth]{../bnbcosmic_output/cnn_85_trackstartZdist.png}
		\caption{Z Vertex Position} 
		\label{fig:cnn85vertexZ}
	\end{subfigure}
\caption{Vertex position for X, Y and Z of true cc-inclusive events passing CNN100000 and Selection I}
\label{fig:vertex}
\end{figure}


\subsection{Classification of MicroBooNE data using Selection I CC-Inclusive Filter}
After checking to make sure CNN100000 truth kinematic distributions looked similar to Selection I, I moved on to use CNN100000 to classify on-beam and off-beam MicroBooNE data. Figures \ref{fig:datatrackrange} through \ref{fig:dataZ} show the different kinematic distributions for on-beam and off-beam data that passed CNN100000 at 80\% muon probability. Comparing BNB+Cosmic truth distributions to MicroBooNE data distributions, the first thing to note is the peaks around $\pm\pi/2$ in the $\phi$ distribution, figure \ref{fig:dataphi} compared to a valley in figure \ref{fig:cnn85phi}. The $\pm\pi/2$ $\phi$ regions, being vertical to the beam direction, are dominated by cosmics. The $\phi$ data distribution points to an excess of cosmics passing CNN100000. 
\begin{figure}[htp!]
\centering
\includegraphics[width=.6\textwidth]{../bnbcosmic_output/data85_trackrangedist.png}
\caption{Track range distribution for on-beam (blue) and off-beam (red) data at CNN10000 $\geq 85\%$} 
\label{fig:datatrackrange}
\end{figure}

\begin{figure}[htp!]
\centering
\includegraphics[width=.6\textwidth]{../bnbcosmic_output/data85_Thetadist.png}
\caption{$Cos(\theta)$ distribution for on-beam (blue) and off-beam (red) data at CNN10000 $\geq 85\%$} 
\label{fig:datatheta}
\end{figure}

\begin{figure}[htp!]
\centering
\includegraphics[width=.6\textwidth]{../bnbcosmic_output/data85_Phidist.png}
\caption{$\phi$ distribution for on-beam (blue) and off-beam (red) data at CNN10000 $\geq 85\%$} 
\label{fig:dataphi}
\end{figure}

\begin{figure}[htp!]
\centering
\includegraphics[width=.6\textwidth]{../bnbcosmic_output/data85_trackstartXdist.png}
\caption{Vertex X distribution for on-beam (blue) and off-beam (red) data at CNN10000 $\geq 85\%$} 
\label{fig:dataX}
\end{figure}

\begin{figure}[htp!]
\centering
\includegraphics[width=.6\textwidth]{../bnbcosmic_output/data85_trackstartYdist.png}
\caption{Vertex Y distribution for on-beam (blue) and off-beam (red) data at CNN10000 $\geq 85\%$} 
\label{fig:dataY}
\end{figure}

\begin{figure}[htp!]
\centering
\includegraphics[width=.6\textwidth]{../bnbcosmic_output/data85_trackstartZdist.png}
\caption{Vertex Z distribution for on-beam (blue) and off-beam (red) data at CNN10000 $\geq 85\%$} 
\label{fig:dataZ}
\end{figure}

To compare MC to data, figures \ref{fig:pottrack} through \ref{fig:potphi} show BNB+Cosmic MC events and off-beam data subtracted from on-beam data that passed CNN100000. On-beam minus off-beam subtracts off the events triggered from cosmics with no beam related interactions which makes it comparable to BNB+Cosmic. The red boxes in the figures correspond to selected $\nu_{\mu}$ CC signal and background. The backgrounds also depicted in various colors as well. From shape, it is visible that there is a decrease in NC background events (green curve) from Selection I to CNN100000, however, there also seems to be a excess in cosmic background events. The cosmic background curve (blue) are cosmic background events in the BNB+Cosmic MC sample that passed a corresponding selection filter. Intime cosmics generated with CORSIKA aren't plotted and are just used for purity calculations. The percentage of NC background events for Selection I was 3.5\%, and for CNN100000 it was 1\%, a decrease of 2.5\%. The CNN is doing a good job at removing NC background while still recovering low track range $\nu_{\mu}$ CC events which was the main goal of this analysis. The percentage of cosmic background from BNB+Cosmic MC dataset for Selection I was 6.6\%, while for CNN100000, it was 23\%, an increase of 16.5\%. This is a large increase from Selection I and CNN100000. One of the reasons this may be is due to the fact that the CNN was trained on single generated isotropic particles, therefore muons from cosmics would in fact pass the CNN even with a high muon probability. Figure \ref{fig:potphi} shows an excess of events in the cosmic enriched sample $\pm\pi/2$, more so than in Selection I. A way to try and combat this would be to add an additional particle class which would consist of muons from cosmics during training to see if the CNN could learn to differentiate beam induced muons from cosmic induces muons. Another way to combat this is to have a more sophisticated flash matching algorithm that could reduce the locality of where the neutrino interaction occurred. You can also see the excess of cosmics in figure \ref{fig:potcostheta}, another way you try and reduce this background is by employing more traditional cuts after the CNN for example cutting on $cos(\theta)$ or $\phi$, but cuts like these affect phase space. 


\begin{figure}[htp!]
\centering
	\begin{subfigure}[t]{.475\textwidth}
		\includegraphics[width=\textwidth]{../bnbcosmic_output/final_pot_cc.png}
		\caption{POT normalized track range distribution plot for Selection I selected $\nu_{\mu}$ CC signal and background as well as off-beam data subtracted by on-beam data.} 
		\label{fig:ccpottrack}
	\end{subfigure}
	\begin{subfigure}[t]{.475\textwidth}
	\centering
		\includegraphics[width=\textwidth]{../bnbcosmic_output/final85_pot.png}
		\caption{POT normalized track range distribution plot for CNN100000 selected $\nu_{\mu}$ CC signal and background as well as off-beam data subtracted by on-beam data.} 
	\label{fig:cnnpottrack}
	\end{subfigure}
\caption{POT normalized track range distributions}
\label{fig:pottrack}
\end{figure}

\begin{figure}[htp!]
\centering
	\begin{subfigure}[t]{.475\textwidth}
		\includegraphics[width=\textwidth]{../bnbcosmic_output/finaleventcosthetadist_pot_cc.png}
		\caption{POT normalized $cos(\theta)$ range distribution plot for Selection I selected $\nu_{\mu}$ CC signal and background as well as off-beam data subtracted by on-beam data.} 
		\label{fig:ccpotcostheta}
	\end{subfigure}
	\begin{subfigure}[t]{.475\textwidth}
	\centering
		\includegraphics[width=\textwidth]{../bnbcosmic_output/finaleventcosthetadist85_pot.png}
		\caption{POT normalized $cos(\theta)$ range distribution plot for CNN100000 selected $\nu_{\mu}$ CC signal and background as well as off-beam data subtracted by on-beam data.} 
	\label{fig:cnnpotcostheta}
	\end{subfigure}
\caption{POT normalized $cos(\theta)$ range distributions}
\label{fig:potcostheta}
\end{figure}

\begin{figure}[htp!]
\centering
	\begin{subfigure}[t]{.475\textwidth}
		\includegraphics[width=\textwidth]{../bnbcosmic_output/finaleventphi_pot_cc.png}
		\caption{POT normalized $\phi$ range distribution plot for Selection I selected $\nu_{\mu}$ CC signal and background as well as off-beam data subtracted by on-beam data.} 
		\label{fig:ccpotphi}
	\end{subfigure}
	\begin{subfigure}[t]{.475\textwidth}
	\centering
		\includegraphics[width=\textwidth]{../bnbcosmic_output/finaleventphi85_pot.png}
		\caption{POT normalized $\phi$ range distribution plot for CNN100000 selected $\nu_{\mu}$ CC signal and background as well as off-beam data subtracted by on-beam data.} 
	\label{fig:cnnpotphi}
	\end{subfigure}
\caption{POT normalized $\phi$ range distributions}
\label{fig:potphi}
\end{figure}

Although in time cosmics aren't shown in figures \ref{fig:pottrack} through \ref{fig:potphi} an interesting thing to note is that there is a decrease in in time cosmic background from Selection I to CNN100000. Selection I had a in time cosmic passing percentage of 33.4\% while CNN100000 had a passing percentage of 5.49\%, a decrease of 27.91\%. For reference, in time cosmic background was scaled to match BNB+Cosmic MC event rate per section \ref{section:normalize}. This explains why there was a large drop in off-beam data that passed CNN100000. 


Another thing that was checked was if the MC/Data difference between the selections were similar. Although it was already stated that CNN100000 was letting in more cosmics from the BNB+Cosmic MC dataset, there was a substantial decrease in in time cosmics. Figure \ref{fig:diff} shows the MC/Data difference for Selection I (red) and CNN100000 (blue). In figure \ref{fig:difftrack} we see the MC/Data difference versus the track range. At higher track ranges there are more MC/Data disagreements from Selection I and CNN100000 than in lower track range bins. This can be attributed to the removal of the track containment cut in CNN100000, which also explains why there are only CNN100000 entries in track ranges greater than 800 cm. When looking at figure \ref{fig:diffcostheta} there are also MC/Data differences between Selection I and CNN100000, at $cos(\theta) = -0.75$ and $ 0.5<cos(\theta)<1$, although these differences are small in comparison to the track range differences. Another thing to note is the fact that the MC/Data difference is negative for figure \ref{fig:diffcostheta}, this is due to there being a data excess for both Selection I and CNN100000. Lastly, in figure \ref{fig:diffphi} you can see an MC/Data difference between Selection I and CNN100000 at the cosmic enriched area around $\phi=\pm\pi/2$. This again is because of the CNN letting more cosmic muons pass than Selection I does. 
\begin{figure}[htp!]
\centering
	\begin{subfigure}[t]{.475\textwidth}
		\includegraphics[width=\textwidth]{../bnbcosmic_output/mc85dataratio_trackrange.png}
		\caption{MC/Data percent difference vs track range for Selection I (red) and CNN100000 (blue)} 
		\label{fig:difftrack}
	\end{subfigure}
	\begin{subfigure}[t]{.475\textwidth}
	\centering
		\includegraphics[width=\textwidth]{../bnbcosmic_output/mc85dataratio_costheta.png}
		\caption{MC/Data percent difference vs $cos(\theta)$ for Selection I (red) and CNN100000 (blue)} 
		\label{fig:diffcostheta}
	\end{subfigure}
	\begin{subfigure}[t]{.475\textwidth}
	\centering
		\includegraphics[width=\textwidth]{../bnbcosmic_output/mc85dataratio_Phi.png}
		\caption{MC/Data percent difference vs $\phi$ for Selection I (red) and CNN100000 (blue)} 
		\label{fig:diffphi}
	\end{subfigure}
\caption{MC/Data percent differences vs kinematic variables}
\label{fig:diff}
\end{figure}
%--------------------------------------Commented out----------------------------------------------
\begin{comment}
\begin{figure}[htp!]
\centering
\includegraphics[width=\textwidth]{../bnbcosmic_output/2d85costhetaphi.png}
\caption{$Cos(\theta)$ distribution at CNN10000 $\geq 85\%$} 
\label{fig:2d85costhetaphi}
\end{figure}

\begin{figure}[htp!]
\centering
\includegraphics[width=\textwidth]{../bnbcosmic_output/2d85trackrangecostheta.png}
\caption{$Cos(\theta)$ distribution at CNN10000 $\geq 85\%$} 
\label{fig:2d85trackrangecostheta}
\end{figure}

\begin{figure}[htp!]
\centering
\includegraphics[width=\textwidth]{../bnbcosmic_output/2d85trackrangephi.png}
\caption{$Cos(\theta)$ distribution at CNN10000 $\geq 85\%$} 
\label{fig:2d85trackrangephi}
\end{figure}
\end{comment}
%--------------------------------------Commented out----------------------------------------------

\newpage

\section{Classification using a modified CNN100000}
To try and reduce the cosmic muon background from the BNB+Cosmic dataset, I reintroduced the track containment cut from the Selection I filter, however, instead of a containment on the longest track, the track with the highest probability of being a muon from an event is chosen and then the track containment is applied. By applying a containment cut, we can possibly reduce the cosmic contamination. Figure \ref{fig:roc_modified} shows the efficiency and purity for various muon probabilities of the modified CNN100000. The efficiency of using a modified CNN100000 is lower than using CNN100000 without containment, however, it is still a higher efficiency than Selection I and there is a vast improvement in the purity compared to both Selection I and Selection II. Comparing a modified CNN100000 to Selection I, at 60\% muon probability, we get an increase of 27\% increase in purity and comparing the CNN to Selection II, a 16\% increase in purity. 

\begin{figure}[htp!]
\centering
\includegraphics[width=.5\textwidth]{../bnbcosmic_output_contained/roc.png}
\caption{Efficiency vs Purity curve for various modified CNN100000 muon probabilities. At 60\% muon probability, the efficiency is 13.9\% and the purity is 81\%} 
\label{fig:roc_modified}
\end{figure}

Figure \ref{fig:truthkinematics_modified} are the true kinematic distributions for the true cc-inclusive events that passed the modified CNN100000 (blue) at 60\% muon probability as well as the cc-inclusive events that passed the Selection I filter (red). The distributions are much more similar by re-introducing the track containment, and we still have a substantial amount of events in the lowest track range bin in figure \ref{fig:cnn60trackrange_modified}.  

\begin{figure}[htp!]
\centering
	\begin{subfigure}[b]{.475\textwidth}
	\centering
		\includegraphics[width=\textwidth]{../bnbcosmic_output_contained/cnn_60_trackrangedist.png}
		\caption{Track range distribution for events passing CNN10000 $\geq 60\%$ and the Selection I filter.} 
		\label{fig:cnn60trackrange_modified}
	\end{subfigure}
	\quad
	\begin{subfigure}[b]{.475\textwidth}
	\centering
		\includegraphics[width=\textwidth]{../bnbcosmic_output_contained/cnn_60_truecosthetadist.png}
		\caption{$Cos(\theta)$ distribution for events passing modified CNN100000 $\geq 60\%$ and the Selection I filter.} 
		\label{fig:cnn60costheta_modified}
	\end{subfigure}
	\quad
	\begin{subfigure}[b]{.475\textwidth}
	\centering
		\includegraphics[width=\textwidth]{../bnbcosmic_output_contained/cnn_60_truephidist.png}
		\caption{$\phi$ distribution for events passing modified CNN100000 $\geq 60\%$ and the Selection I filter.} 
		\label{fig:cnn60phi_modified}
	\end{subfigure}
	\quad
	\begin{subfigure}[b]{.475\textwidth}
	\centering
		\includegraphics[width=\textwidth]{../bnbcosmic_output_contained/cnn_60_momentumdist.png}
		\caption{Momentum distribution for events passing modified CNN100000 $\geq 60 \%$ and the Selection I filter.} 
		\label{fig:cnn60momentum_modified}
	\end{subfigure}
\caption{Truth kinematic distributions of events passing modified CNN100000 and Selection I. The red corresponds to the Selection I passing events and blue to the modified CNN100000 passing events.}
\label{fig:truthkinematics_modified}
\end{figure}

Figures \ref{fig:trackstacked_modified} through \ref{fig:phistacked_modified} compare the stacked event type distributions between the Selection I filter and the modified CNN100000. The percentage of CCQE events that passed Selection I is 51\% , 62\% for CNN100000 and 47\% for modified CNN100000. For CCRES, Selection I had a passing rate of 37\%, 29\% for CNN100000 and 40\% for modified CNN100000. The CCDIS passing rate was 11\%, 9\%, and 12\% for Selection I, CNN100000, and modified CNN100000 respectively. Lastly, the CCCOH rate was 1\%, 0.9\%, and 1\% for Selection I, CNN100000, and modified CNN100000 respectively. The passing rates per interaction type for Selection I and modified CNN100000 are now much more comparable.  

\begin{figure}[htp!]
\centering
	\begin{subfigure}[b]{.475\textwidth}
	\centering
		\includegraphics[width=\textwidth]{../bnbcosmic_output_contained/cc_trackrangedist_stacked.png}
		\caption{Track range distribution for events passing Selection I filter.} 
		\label{fig:cctrackstacked_modified}
	\end{subfigure}
	\quad
	\begin{subfigure}[b]{.475\textwidth}
	\centering
		\includegraphics[width=\textwidth]{../bnbcosmic_output_contained/cnn_60_trackrangedist_stacked.png}
		\caption{Track range distribution for events passing modified CNN100000 $\geq 60\%$.} 
		\label{fig:cnn60trackstacked_modified}
	\end{subfigure}
\caption{Truth stacked event type track range distribution of events passing Selection I (left) and modified CNN100000 (right). Different event types are CCQE (green), CCRES (blue), CCDIS (yellow), CCCOH (red).}
\label{fig:trackstacked_modified}
\end{figure}

\begin{figure}[htp!]
\centering
	\begin{subfigure}[b]{.475\textwidth}
	\centering
		\includegraphics[width=\textwidth]{../bnbcosmic_output_contained/cc_momentumdist_stacked.png}
		\caption{Momentum distribution for events passing Selection I.} 
		\label{fig:ccmomentumstacked_modified}
	\end{subfigure}
	\quad
	\begin{subfigure}[b]{.475\textwidth}
	\centering
		\includegraphics[width=\textwidth]{../bnbcosmic_output_contained/cnn_60_momentumdist_stacked.png}
		\caption{Momentum distribution for events passing modified CNN100000 $\geq 60\%$.} 
		\label{fig:cnn60momentumstacked_modified}
	\end{subfigure}
\caption{Truth stacked event type momentum distribution of events passing Selection I (left) and modified CNN100000 (right). Different event types are CCQE (green), CCRES (blue), CCDIS (yellow), CCCOH (red).}
\label{fig:momentumstacked_modified}
\end{figure}

\begin{figure}[htp!]
\centering
	\begin{subfigure}[b]{.475\textwidth}
	\centering
		\includegraphics[width=\textwidth]{../bnbcosmic_output_contained/cc_costhetadist_stacked.png}
		\caption{$Cos(\theta)$ distribution for events passing Selection I filter.} 
		\label{fig:cccosthetastacked_modified}
	\end{subfigure}
	\quad
	\begin{subfigure}[b]{.475\textwidth}
	\centering
		\includegraphics[width=\textwidth]{../bnbcosmic_output_contained/cnn_60_costhetadist_stacked.png}
		\caption{$Cos(\theta)$ distribution for events passing modified CNN100000 $\geq 60\%$.} 
		\label{fig:cnn60costhetastacked_modified}
	\end{subfigure}
\caption{Truth stacked event type $cos(\theta)$ distribution of events passing Selection I (left) and modified CNN100000 (right). Different event types are CCQE (green), CCRES (blue), CCDIS (yellow), CCCOH (red).}
\label{fig:costhetastacked_modified}
\end{figure}

\begin{figure}[htp!]
\centering
	\begin{subfigure}[b]{.475\textwidth}
	\centering
		\includegraphics[width=\textwidth]{../bnbcosmic_output_contained/cc_phidist_stacked.png}
		\caption{$\phi$ distribution for events passing Selection I.} 
		\label{fig:ccphistacked_modified}
	\end{subfigure}
	\quad
	\begin{subfigure}[b]{.475\textwidth}
	\centering
		\includegraphics[width=\textwidth]{../bnbcosmic_output_contained/cnn_60_phidist_stacked.png}
		\caption{$\phi$ distribution for events passing modified modified CNN100000 $\geq 60\%$.} 
		\label{fig:cnn60phistacked_modified}
	\end{subfigure}
\caption{Truth stacked event type phi distribution of events passing Selection I (left) and modified modified CNN100000 (right). Different event types are CCQE (green), CCRES (blue), CCDIS (yellow), CCCOH (red).}
\label{fig:phistacked_modified}
\end{figure}


The final event distributions are shown in figures \ref{fig:pottrack_modified} through \ref{fig:potphi_modified}. There is still an MC/Data difference. The BNB+Cosmic cosmics passing the modified CNN100000 is 6\%, a reduction of 17\% from the CNN100000 without containment and only a 0.6\% difference from Selection I. The NC passing percentage for modified CNN100000 is 6\%, higher than both the un-contained CNN and Selection I. Although the cosmic muon rate is reduced by re-introducing the track containment, reducing the muon probability allows for NC pions to seep in to the selection. Also, reducing the muon probability is necessary because a containment cut affects efficiency so to recover a reasonable efficiency, the muon probability cut needs to be less strict. The in time cosmics passing rate for the modified CNN100000 is 6\%, comparable to the un-contained CNN100000 and lower than Selection I by 27\%. 
\begin{figure}[htp!]
\centering
	\begin{subfigure}[t]{.475\textwidth}
		\includegraphics[width=\textwidth]{../bnbcosmic_output_contained/final_pot_cc.png}
		\caption{POT normalized track range distribution plot for Selection I selected $\nu_{\mu}$ CC signal and background as well as off-beam data subtracted by on-beam data.} 
		\label{fig:ccpottrack_modified}
	\end{subfigure}
	\begin{subfigure}[t]{.475\textwidth}
	\centering
		\includegraphics[width=\textwidth]{../bnbcosmic_output_contained/final60_pot.png}
		\caption{POT normalized track range distribution plot for modified CNN100000 selected $\nu_{\mu}$ CC signal and background as well as off-beam data subtracted by on-beam data.} 
	\label{fig:cnnpottrack_modified}
	\end{subfigure}
\caption{POT normalized track range distributions}
\label{fig:pottrack_modified}
\end{figure}

\begin{figure}[htp!]
\centering
	\begin{subfigure}[t]{.475\textwidth}
		\includegraphics[width=\textwidth]{../bnbcosmic_output_contained/finaleventcosthetadist_pot_cc.png}
		\caption{POT normalized $cos(\theta)$ range distribution plot for Selection I selected $\nu_{\mu}$ CC signal and background as well as off-beam data subtracted by on-beam data.} 
		\label{fig:ccpotcostheta_modified}
	\end{subfigure}
	\begin{subfigure}[t]{.475\textwidth}
	\centering
		\includegraphics[width=\textwidth]{../bnbcosmic_output_contained/finaleventcosthetadist60_pot.png}
		\caption{POT normalized $cos(\theta)$ range distribution plot for modified CNN100000 selected $\nu_{\mu}$ CC signal and background as well as off-beam data subtracted by on-beam data.} 
	\label{fig:cnnpotcostheta_modified}
	\end{subfigure}
\caption{POT normalized $cos(\theta)$ range distributions}
\label{fig:potcostheta_modified}
\end{figure}

\begin{figure}[htp!]
\centering
	\begin{subfigure}[t]{.475\textwidth}
		\includegraphics[width=\textwidth]{../bnbcosmic_output_contained/finaleventphi_pot_cc.png}
		\caption{POT normalized $\phi$ range distribution plot for Selection I selected $\nu_{\mu}$ CC signal and background as well as off-beam data subtracted by on-beam data.} 
		\label{fig:ccpotphi_modified}
	\end{subfigure}
	\begin{subfigure}[t]{.475\textwidth}
	\centering
		\includegraphics[width=\textwidth]{../bnbcosmic_output_contained/finaleventphi60_pot.png}
		\caption{POT normalized $\phi$ range distribution plot for modified CNN100000 selected $\nu_{\mu}$ CC signal and background as well as off-beam data subtracted by on-beam data.} 
	\label{fig:cnnpotphi_modified}
	\end{subfigure}
\caption{POT normalized $\phi$ range distributions}
\label{fig:potphi_modified}
\end{figure}


Figure \ref{fig:diff_modified} shows the MC/Data difference for Selection I (red) and modified CNN100000 (blue). In figure \ref{fig:difftrack_modified} we see the MC/Data difference versus the track range. There are no differences from Selection I and modified CNN100000 other than the lack of entries for Selection I in the lower track range bin. When looking at figure \ref{fig:diffcostheta_modified} there are small MC/Data differences between Selection I and modified CNN100000, at $ 0.5<cos(\theta)<0.5$, although these differences fall within statistical uncertainties. Another thing to note is the fact that the MC/Data difference is negative for figure \ref{fig:diffcostheta_modified}, this is due to there being a data excess for both Selection I and modified CNN100000. Lastly, in figure \ref{fig:diffphi_modified} you can see an MC/Data difference between Selection I and modified CNN100000 at the cosmic enriched area around $\phi=\pm\pi/2$. Modified CNN100000 is still letting in more cosmics than Selection I, although at a much smaller rate than the original CNN100000. 
\begin{figure}[htp!]
\centering
	\begin{subfigure}[t]{.475\textwidth}
		\includegraphics[width=\textwidth]{../bnbcosmic_output_contained/mc60dataratio_trackrange.png}
		\caption{MC/Data percent difference vs track range for Selection I (red) and modified CNN100000 (blue)} 
		\label{fig:difftrack_modified}
	\end{subfigure}
	\begin{subfigure}[t]{.475\textwidth}
	\centering
		\includegraphics[width=\textwidth]{../bnbcosmic_output_contained/mc60dataratio_costheta.png}
		\caption{MC/Data percent difference vs $cos(\theta)$ for Selection I (red) and modified CNN100000 (blue)} 
		\label{fig:diffcostheta_modified}
	\end{subfigure}
	\begin{subfigure}[t]{.475\textwidth}
	\centering
		\includegraphics[width=\textwidth]{../bnbcosmic_output_contained/mc60dataratio_Phi.png}
		\caption{MC/Data percent difference vs $\phi$ for Selection I (red) and modified CNN100000 (blue)} 
		\label{fig:diffphi_modified}
	\end{subfigure}
\caption{MC/Data percent differences vs kinematic variables}
\label{fig:diff_modified}
\end{figure}



\section{Calculating preliminary cc-inclusive cross-sections for each selection}
For reference purposes, the cc-inclusive cross-sections were calculated for Selection I, Selection I, CNN100000, and modified CNN100000. The cc-inclusive equation is shown in \ref{eq:crosssection}.
\begin{equation}
\sigma = \frac{N_{meas}-N_{Bkg}}{\epsilon*N_{target}*\Phi_{\nu_{\mu}}}
\label{eq:crosssection}
\end{equation}
where:
\begin{conditions}
N_{meas} & on-beam data\\
N_{Bkg} & off-beam data\\
\epsilon & efficiency of selection\\
N_{target} & number of target nucleons in fiducial volume\\
\Phi_{\nu_{\mu}} & BNB $\nu_{\mu}$ flux integrated over $E_{\nu_{\mu}}$ and scaled to corresponding POT used in selection\\
\end{conditions}
The on-beam data and off-beam data passing the four selections are shown in table \ref{table:datapassingrates} as well as the efficiencies.

\begin{table}[htp!]
\centering
\resizebox{.6\textwidth}{!}{ \begin{tabular}{c||c c c} 
\hline %inserts double horizontal lines
& On-Beam Data & Off-Beam Data & Efficiency\\
\hline %inserts double horizontal lines
Selection I & 3213 & 1328 & 12.3\% \\
Selection II & 3228 & 528 & 28.7\% \\
CNN100000 & 7606 & 1401 & 30\% \\
CNN100000 modified & 2569 & 296 & 13.9\% \\
\end{tabular}}
\caption{}
\label{table:datapassingrates}
\end{table}
To calculate $N_{target}$, the fiducial volume is necessary. For all the selections, the fiducial volume is 10 cm subtracted from edges in X (drift direction) and Z (beam direction) and 20 cm subtracted from edges in Y (vertical direction) leaving the XYZ values to be:
\begin{conditions}
X & 236.35 cm\\
Y & 193 cm\\
Z & 1016.8 cm\\
\end{conditions}

therefore $V_{fid}$ = $46.6E^6 \text{cm}^3$. Using the equation \ref{eq:ntarget} $N_{target}$ = $3.917E^{31}$. 

\begin{equation}
N_{target} = V_{fid}*\frac{\rho_{Ar}}{M_{Ar}}*\#_{\text{Ar nucleons}}
\label{eq:ntarget}
\end{equation}
where:
\begin{conditions}
V_{fid} & 46.6E^{6}\\
\rho_{Ar} & 1.4 g/cm^{3}\\
M_{Ar} & 6.63E^{-23} g\\
\#_{\text{Ar nucleons}} & 40\\
\end{conditions}

The BNB $\nu_{\mu}$ flux is plotted in figure \ref{fig:xsecflux}. The flux was then integrated over all neutrino energy and scaled to 5e19 POT and was found to be $3.555E^{10}$.
\begin{figure}[htp!]
\centering
\includegraphics[width=.6\textwidth]{figs/bnbflux.pdf}
\caption{BNB $\nu_{\mu}$ flux versus $E_{\nu}$ scaled to 5e19 POT. The black line is the mean neutrino energy with the red dotted lines showing $1\sigma$ energy range. For plotting cross-section on world data plot, the mean neutrino energy is at $679.3^{+545}_{-543}$}
\label{fig:xsecflux}
\end{figure}
The calculated cross-sections are then:

\begin{equation}
\sigma_{\text{selI}} = \frac{3213-1328}{.123*3.917E^{31}*3.555E^{10}} = 1.1E^{-38}\\
\sigma_{\text{selII}} = \frac{3228-528}{.287*3.917E^{31}*3.555E^{10}} = 0.68E^{-38}\\
\sigma_{\text{CNN100000}} = \frac{7606-1401}{.30*3.917E^{31}*3.555E^{10}} = 1.48E^{-38}\\
\sigma_{\text{CNN100000 modified}} = \frac{2569-296}{.139*3.917E^{31}*3.555E^{10}} = 1.17E^{-38}\\
\end{equation}
These selection cross-sections are plotted in figure \ref{fig:xsec} with the rest of the cross-section world data. Statistical and systematic errors still need to be calculated, but these datapoints are the starting point and give an idea of each of the selection's potential cross-section compared to the world data.

\begin{figure}[htp!]
\centering
\includegraphics[width=.8\textwidth]{figs/c_data_PDB_02_contained.png}
\caption{World data of neutrino cross-section measurements with MicroBooNE Selection I (pink), Selection II (cyan), CNN100000 (blue), and modified CNN100000 (green) datapoints.}
\label{fig:xsec}
\end{figure}

  %\chapter{Using Convolutional Neural Networks on MicroBooNE Data}\label{ch:data}
\dots 
\clearpage
\dots

  %\chapter{Comparing two CC-Inclusive Cross Section Selection Filters}
\dots
\clearpage
\dots

  \chapter{Conclusion}
%\addcontentsline{toc}{chapter}{Conclusion}
Neutrinos, specifically neutrino oscillations, can possibly answer questions beyond the standard model. Using neutrinos to probe physics beyond the standard model however requires a thorough understanding of $\nu-N$ interactions. CC-inclusive cross-sections allow us to understand these interactions as well as help in reconstructing initial neutrino energies which is needed for neutrino oscillation studies. MicroBooNE will be the first $\nu-Ar$ cross-section measurement in the 1 GeV range, where CCQE interactions dominate. The work done in this thesis was to improve the existing cc-inclusive event selection to encompass as much of the $E_{\nu}$ spectra in the 1 GeV range that MicroBooNE sees. 

A review of neutrinos, neutrino oscillations, and neutrino interactions was discussed as well as LArTPCs, specifically MicroBooNE. A description of the first analysis using MicroBooNE data was then outlined. The neutrino ID analysis was the first time automated reconstruction was used on LArTPC data to find neutrino events. The reconstruction was tuned using BNB+Cosmic events and similar topological cuts were then used to build a cc-inclusive selection filter for use in finding cc-inclusive events for the cross-section measurement. The cc-inclusive selection filter to date was then described in detail as well as the efficiency and purity.  

A background of convolutional neural networks was discussed as well as the hardware frameworks and CNN architectures used for training. The image making scheme used in this analysis was described in detail and results of multiple CNN trainings were outlined. CNNs were then used to search for cc-inclusive events in MC and data and the results of this search were described. 

This analysis was successful in training a convolutional neural network on LArTPC data. This alone is pioneering in the field of high energy physics. The analysis was also successful in using a single generated particle trained CNN to classify BNB+Cosmic MC and data events. Successes also include improving both efficiency and purity of all previous cc-inclusive selection filters without affecting truth kinematic distributions. 

Lastly, the analysis was successful in using CNNs for $\mu/\pi$ separation, reducing the NC background, as well as recovering cc-inclusive events below the previously imposed 75 cm track range cut. By removing the 75 cm track range cut, we allow for improvements to the cc-inclusive cross-section measurement MicroBooNE can measure, especially with the increase in CCQE events that can lead to more precise CCQE cross-section measurements that can then help the neutrino oscillation measurement on MicroBooNE. An understanding of the irreducible backgrounds in the CNN100000 selection (specifically cosmics), the MC/Data difference and the statistical and systematic errors are still left to do before a cc-inclusive cross-section in MicroBooNE is finished, however, large strides were made in this analysis to improve the selection.  



  %% To ignore a specific chapter while working on another, making the build faster, comment it out:
  %\input{chap4}
\end{mainmatter}


%% Produce the un-numbered back matter (e.g. colophon,
%% bibliography, tables of figures etc., index...)
\begin{backmatter}
  \input{backmatter}
\end{backmatter}

%% Produce the appendices
\begin{appendices}
  %% The "\appendix" call has already been made in the declaration
%% of the "appendices" environment (see thesis.tex).
\chapter{Curriculum Vitae}

\includepdf[pages=-]{../post\_doc\_applications/cv\_7/esquivel\_cv\_wpublications.pdf}


\end{appendices}

%% Close
\end{document}
