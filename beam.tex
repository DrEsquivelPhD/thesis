\chapter{The Booster Neutrino Beam}\label{ch:beam}

The MicroBooNE detector is stationed at Fermi National Accelerator Laboratory (FNAL) where it receives neutrinos from both the Booster Neutrino Beam (BNB) and Neutrinos from the Main Injector (NuMI) beams. MicroBooNE is on-axis for the BNB and off-axis by 135 mrad for NuMI. For the purpose of this analysis, only data from the BNB was used. This chapter will discuss how neutrinos are created using the BNB. How these neutrinos are produced as well as their flux through the MicroBooNE detector is necessary for any analysis because of the systematic uncertainties the beam introduces to a measurement.

\section{Creating the Booster Neutrino Beam}
The BNB is a very pure $\nu_{\mu}$ beam, with only 0.6\% contamination from $\nu_{e}s$. The energy also peaks around 700 MeV which is desired based on the probability of oscillation equation which depends on the the value of $L/E$, where $L$ is the distance of the detector from the neutrino beam and $E$ is the energy of the neutrino beam. $L/E$ was chosen to increase the probability of seeing neutrino oscillations in the MiniBooNE Low Energy Excess (LEE) range based on the probability of oscillation equation, which is $ P_{\nu_{\mu}\rightarrow \nu_{e}}\left(L,E\right) = \sin^2 2\theta \sin^2 \left(1.27\Delta m^2 \frac{L}{E_{\nu}}\right)$. The BNB collides 8.9 GeV/c momentum protons from the FNAL booster synchrotron into a beryllium target which produces a high flux of neutrinos. The protons originate from $H^2$ gas molecules that are turned into $H^-$ ions by a Cockroft-Walton generator shown in figure \ref{fig:generator}. The $H^-$ initially are accelerated to 1MeV kinetic energy and are then passed to a linear accelerator using alternating electromagnetic fields to increase their energy to ~400MeV. The ions are stripped of electrons by passing them through a carbon foil. The protons are bunched into beam spills which contain ~$4*10^12$ protons in a 1.6 $\micro s$ time window per spill. It's at this point that the protons are directed towards the beryllium target. The amount of protons directed towards the target (POT) is measured by two toroids upstream of the target with an error of ~2$\%$. Beam intensity, timing, width, position, and direction are monitored by beam position monitors, multi-wire chamber and resistive monitors.    





\begin{figure}[htp!]
\centering
	\begin{subfigure}[b]{.4\textwidth}
	\includegraphics[width=\textwidth]{figs/lee.png}
	\caption{Low Energy excess seen in MiniBooNE}
	\label{fig:lee}
	\end{subfigure}
	\quad
	\begin{subfigure}[b]{.4\textwidth}
	\includegraphics[width=\textwidth]{figs/bnbflux.png}
	\caption{Energy spectrum of the Booster Neutrino Beam at Fermi National Laboratories}
	\label{fig:bnbflux}
	\end{subfigure}
	\quad
\label{fig:figures}
\caption{\ref{fig:bnbflux} Flux of BNB at FNAL.}
\end{figure}
