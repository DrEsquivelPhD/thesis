\chapter{The MicroBooNE Experiment}\label{ch:microboone}
The purpose of this chapter is to discuss and understand the details of the MicroBooNE detector. A thorough understanding of MicroBooNE and the technology behind liquid argon time projection chambers is important for understanding results as well as understanding how images were made for use in deep learning efforts that will be outlined in later chapters.   

\section{Liquid argon time projection chambers}
Liquid Argon Time Projection Chambers (LArTPCs) are an exciting detector technology that provide excellent imaging and particle identification, and are now being used to study neutrinos. The Time Projection Chamber (TPC) was first invented by Nygren in 1974 \cite{nygren} and the proposal for a LArTPC for neutrino physics was made by Rubbia \cite{rubbia} in 1977 with the ICARUS collaboration implementing this concept\cite{icaraus}. A LArTPC is a three-dimensional imaging detector that uses planes of wires at the edge of an active volume to read out an interaction. When a neutrino interacts with an argon atom, the charged particles that are produced ionize the LAr as they travel away from the interaction. By placing a uniform electric field throughout the LAr volume, the ionization is made to drift towards a set of anode planes, which consist of wires spaced very closely together collecting the ionized charge, which is subsequently read out by electronics connected to the anode wires. The collected ionization creates a spatial image of what happened in the detector on each anode plane. The position resolution of the interaction along the beam direction (perpendicular to drift direction) relies on the wire pitch, while the resolution in drift direction is dependent on the timing resolution of the electronics used and the longitudinal diffusion in the volume. The drift time of the ionization relative to the time of the original signal allows the signal to be projected back along the drift coordinate, hence the name LArTPC. Having very small distances between each wire within an anode plane allows for very fine granularity and detail to be captured, and having multiple wire planes at different angles provides independent two-dimensional views that can be combined into a three dimensional picture of the interaction. Once the charge signal is created on the anode planes, software analysis packages identify particles in the detector by using deposited energy on the wires along their track length. 
The 30 year development of the ICARUS detector has led to LArTPCs being used as cosmic ray \cite{lartpc_cosmic}, solar neutrino \cite{lartpc_solar} and accelerator neutrino \cite{lartpc_accelerator} detectors. The ArgoNeuT experiment at Fermilab was the first United States based liquid argon neutrino program that has since produced short-baseline $\nu-Ar$ cross-section measurements in the NUMI beamline \cite{argoneut}. The MicroBooNE experiment is the second experiment in the US based LArTPC neutrino program and will be discussed thoroughly in the next sections.  
The next phases of the liquid argon neutrino program are under way and are the Fermilab Short Baseline Neutrino (SBN) program \cite{sbn} and the Deep Underground Neutrino Experiment (DUNE) \cite{dune}. The SBN program will include three LArTPC detectors, including the MicroBooNE detector, on the Booster Neutrino Beam (BNB) to do multiple-baseline oscillation measurements. The detector closest to the beam will be the 40 ton Short Baseline Neutrino Detector (SBND)\cite{sbnd} at 150 m and the detector furthest is the 600 ton ICARUS T600 \cite{icarus_t600} detector positioned at 600 m. The DUNE collaboration will deliver a 30 GeV neutrino beam 1300 km from Fermilab to a 34 kiloton LArTPC detector at Homestake, SD. DUNE will study the leptonic CP phase, $\delta_{cp}$, as well as measure neutrino and antineutrino oscillations. 
\section{The MicroBooNE Time Projection Chamber}
MicroBooNE (Micro Booster Neutrino Experiment) is a ~89 ton active volume (180 ton total mass) LArTPC which is then inserted into a cylindrical crysotat on axis of the Booster Neutrino Beam (BNB) stationed at Fermilab in Batavia, Illinois. Understanding LArTPC technology and detector physics is necessary to build a LArTPC the size of DUNE, and MicroBooNE has made many advances in developing this technology\cite{noisechar} \cite{michel}. 

MicroBooNE's Time Projection Chamber (TPC) is 10.3 m long (beamline direction), 2.3 m high and 2.5 m wide (which corresponds to the drift distance). The TPC is shown in figure \ref{fig:tpc}. MicroBooNE is the largest LArTPC currently running in the world\cite{microboone}. This LArTPC has 3 wire planes: 1 plane that collects the ionization in the wires and is $0^{\circ}$ to the virtical with 3456 wires spaced 3 mm apart, and 2 planes where the ionization drifts passed and induces a signal at $\pm 60^{\circ}$ to the vertical each with 2400 wires also spaced 3 mm apart. Each plane has a spacing also of 3 mm from eachother. The first two planes are the induction planes and the last is the collection. The 270 V/cm electric field of the TPC is created using 64 stainless steel tubes shaped into rectangules around the TPC and held in place by G10 to form a field cage. The cathode is charged at a high voltage of -70 kV and this voltage is stepped down across the field cage tubes using a voltage divider chain with an equivalent resitance of 240 $M\ohm$ between the tubes. The field cage tubes are separated by 4 cm from center to center. The electron drift distance is 2.5 m in the x direction with a drift time of 2.3 ms. Maintaining high charge yield is done by continuously recirculating and purifying the argon. The purity is monitored using MicroBooNE's light collection system. Another use of the light collection system is initial timing and drift coordinate of the interaction. 

MicroBooNE's light collection system is a crucial part for 3D reconstruction of all particle interactions in the LArTPC. The initial interaction time, $t_0$, and initial drift coordinate, $x_0$, are not known from the TPC alone. For beam events, the accelerator clock is used to determine $t_0$ of the interaction and the $x_0$ can be inferred using drift time. Non-beam events, however, do not have this capability, which is why scintillation light from an interaction is used. The $\nu-Ar$ interaction produces scintillation light which is collected by photomultiplier tubes (PMTs) which allows the exact time, $t_0$ of the neutrino interaction to be determined. The scintillation light created propogates within nanoseconds to the light collection system compared to the milliseconds it takes the ionized electrons from the interaction to reach the anode wire planes. Therefore we can precisely know where along the drift direction the particle interaction first took place. The scintillation light is also localized, so combining the PMT information with the wire plane information allows for cosmic background rejection happening outside the beam timing window.  

The light collection system is made up of 32 Hammamatsu R5912-02mod cryogenic PMTs with a diameter of 8-inches. The PMTs are located behind the 3 wire anode planes and provides 0.85\% photocathode coverage. Each PMT has an acrylic plate mounted in front of it that is coated with a wave-length shifting material called TPB. The acrylic plates take in the scintillation light, at 128 nm, and re-emits it visible wavelengths visible to the PMTs, with a peak at 425 nm. 

Both the light collection system and the TPC create analog signal that is read out and digitized by the electronics system. The process requires amplification and shaping of the signal which then is goes to the data aquisition (DAQ) software for writing of the digitized data to disk. The anode plane wires are connected to detector specific circuit boards (ASICS) that are submerged and operate inside the liquid argon volume. These ASICS send amplified signal to 11 feed-throughs where further amplification of the signal happens outside the cryostat. The signal is received by custom LArTPC readout modules distributed over nine readout crates which do the digitization. The TPC wires are digitized at 16 MHz then downsampled to 2 MHz. The TPC system reads out 4 frames of wire signal data per event, 1 frame before a trigger and 2 frames after the triggered frame. The four frames allows for identification of a neutrino interaction as well as cosmic background rejection. The process of digitization is similar for the light collection system. Each PMT signal undergoes a shaping with a 60 ns  peaking time for digitization of multiple samples. The digitization occurs at 64 MHz but are not read out continuously during the TPC readout time. Only shaped PMT signal samples above a small threshold are read out and saved. Both the TPC and PMT readouts are initiated via triggers on a separate trigger board located in a warm electronics crate. The timing trigger is created by a timing signal from the BNB accelerator which is shaped and sent to the trigger board. The PMT trigger is generated when the PMT signal multiplicity is greater than 1 and the summed PMT pulse-height is more than 2 photo-electrons summed up over all PMT channels. When the trigger board gets both a timing trigger and a PMT trigger in coincidence, at BNB trigger is then generated by the board. This signal is then passed to all readout crates intiating the readout of data. The data is then sent to the DAQ software which then saves the data to disk into one event memory.

\section{MicroBooNE's Physics Goals} 
\subsection{The low-energy excess}
The primary goal of the MicroBooNE experiment is to study and investigate the low-energy excess seen in MiniBooNE. MicroBooNE has the capability of confirming or denying this excess as electrons or photons due to the detector being in the same beam, having a similar baseline, and lastly the detector being able to clearly distinguish between electrons and photons. LArTPCs use the topology of events as well as energy loss near the vertex to differentiate petween single $e^-$ tracks and photon-induced induced pair production $\gamma \rightarrow e^+e^-$, which wasn't possible in MiniBooNE, a Cherenkiv detector. This technique has been shown in the ArogoNeuT detector\cite{argoneut} and a side by side comparison of both event types in a LArTPC can be seen in figure \ref{fig:egamma}. An excess in electrons would point towards new oscillation physics beyond the standard model, while photons would be within the standard model. MicroBooNE will ovserve a 4-5$\sigma$ signal.  
\subsection{Cross sections}
MicroBooNE's neutrino cross-section program will be the first $\nu-Ar$ cross-section in the 1 GeV energy range and one of only a few cross-section measurements of $\nu-Ar$ in the world. MicroBooNE is also the first liquid argon detector to collect the highest statistics sample of neutrino interactions. Investigating final-state-interactions in the 1GeV energy range provides information about short range nuclear correlations that affect the interpretations of neutrino oscillation experiment data. 
\subsection{Astroparticle physics}
$99\%$ of energy leaving a supernova leaves in the form of neutrinos which can be seen by detectors on earth's surface. MicroBooNE will have a dedicated supernova data stream and a connection to the SuperNova Early Warning System \cite{snews} so when a nearby supernova explosion occurs, continuous data will be written for several hours that can be later analyzed for supernova neutrino events. This information coupled with information from other experiments can be used to better understand supernovae.\textcolor{red}{\textbf{add infograph of supernova here}}
\subsection{Liquid argon detector development}
The last physics goal for the MicroBooNE collaboration is to provide important information regarding LArTPC technology. Being the first in large scare LArTPCs in the US, MicroBooNE will be albe to provide improvements to High Voltage (HV) distribution, Noise Characterization \cite{noise}, and Michel Electron Reconstruction \cite{michel}. 
