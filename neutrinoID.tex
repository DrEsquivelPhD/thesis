\chapter{Neutrino Identification: Finding MicroBooNE's first Neutrinos} \label{ch:neutrinoID}
The goal of the Neutrino Identification analysis was to positively identify BNB neutrino interactions in the MicroBooNE detector collected during the first days of running. Neutrino event candidates were identified in part by using a cut on detected flash of scintillation light during the 1.6 $\micro s$ beam-spill length of the BNB as well as identifying reconstructed object from the TPC that are neutrino like. After this selection, 2D and 3D event displays were used for verification of the selection performance. This selection was targeted to reduce the ratio of neutrino events to cosmic-only events from the initial 1 neutrino to 675 cosmics to a ratio of 1 to 0.5 or better which is equivalent to a background reduction by a factor of 1000 or more. These selected events were used for MicroBooNE's public displays of neutrino interactions. A clearly visible neutrino interaction with an identifiable vertex and at least 2 tracks originating from the vertex was what the analysis focused on. This analysis wasn't optimized for high purity or efficiency, but rather for very distinguishable neutrino interactions that could be identified by the public.
\section{} 
